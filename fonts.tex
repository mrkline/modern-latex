\chapter{Fonts}
\label{fonts}

Digital type has changed almost entirely since \LaTeX{} was released in the
1980s.
\LaTeX{} originally used used \MF, a system designed by Donald Knuth specifically
for \TeX{}.
As time went on, support for PostScript\footnote{One of
Adobe's original claims to fame,
PostScript is a language for defining and drawing computer graphics,
including type. It remains in use today.} fonts was added
as they grew in popularity.
Today, \LuaLaTeX{} and \XeLaTeX{} offer support for the two formats you're
likely to encounter on your computer:
TrueType and OpenType.

\begin{description}
\item[TrueType] was developed by Apple and Microsoft in the late 1980s.
    Most of the fonts that come pre-installed on your system are likely
    in this format.
    TrueType files generally end in a \monobox{.ttf} extension.
\item[OpenType] was first released in 1996 by Microsoft and Adobe.
    One of its major improvements over TrueType is the ability to embed
    multiple versions of glyphs in a single font file.
    We'll see this feature in action through much of this chapter.
    OpenType files generally end in an \monobox{.otf} extension.
\end{description}

\section{Changing typefaces}

By default, \LuaLaTeX{} and \XeLaTeX{} use Latin Modern,
an OpenType rendition of \LaTeX's original type family, Computer Modern.
While it is a high-quality family of typefaces,
you may want to use different ones for your document.
This is done through the \texttt{fontspec} package:
