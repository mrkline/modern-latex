\chapter{Hello, \texorpdfstring{\LaTeX}{LaTeX}!}

Now that you have a \LaTeX{} distribution successfully installed,
let's try it out.
Open up your editor of choice and save the following as \texttt{hello.tex}:
\begin{leftfigure}
\begin{lstlisting}
\documentclass{article}
% Say hello
\begin{document}
Hello, World!
\end{document}
\end{lstlisting}
\end{leftfigure}
We now want to run this through a \LaTeX{} program,
also called an \introduce{engine}, to get our document.
The installation will have placed several different engines on your
machine, but for this entire book, we'll always use \LuaLaTeX or \XeLaTeX.
These are the newest engines available---see Appendix~\ref{history} for an
explanation of how the various versions of \LaTeX{} differ.

If you are using a \LaTeX{}-specific editor, it may contain a drop-down menu
or some other configuation to select the version you'd like to use,
and a button to generate your document.
Otherwise, from a terminal,\punckern\footnote{How to work a terminal emulator,
making sure the newly-installed \LaTeX{} programs are in your \texttt{\$PATH},
and so on are all outside the scope of this book.
As is tradition, the leading dollar signs in examples just denote a console
prompt, and shouldn't actually be typed.}
run the following:
\begin{leftfigure}
\begin{lstlisting}
$ xelatex -halt-on-error hello.tex
\end{lstlisting}
\end{leftfigure}
Feel free to use \texttt{lualatex} instead---there are a few differences
between the two, but either is fine for now.
With luck, you should see some output, ending in a message similar to:
\begin{leftfigure}
\begin{lstlisting}
Output written on hello.pdf (1 page).
Transcript written on hello.log.
\end{lstlisting}
\end{leftfigure}
and in the same directory, a newly minted \texttt{hello.pdf}.
Open it up and you should find a page with this at the top:
\begin{leftfigure}
\lm Hello, World!
\end{leftfigure}
Congrats!
You just created your first document with \LaTeX.
Now let's unpack what we did.

All \LaTeX{} documents begin with a \verb|\documentclass| declaration,
which picks a style to use.
Many classes are available---and you can even create your own---but common
ones include \texttt{article}, \texttt{report}, \texttt{book},
and \texttt{beamer}.\punckern\footnote{This last one is for slideshows.
Kind of an odd name, no?}
For your average document, \texttt{article} is probably a good choice.
On our next line, we had \verb|% Say hello|.
This is a \introduce{comment}---anything placed after a percent symbol on a line
will be ignored by the engine.
Like comments in programming,
you should use these to explain what's going on to anybody reading the
document.\punckern\footnote{Including, and perhaps most importantly,
a later version of yourself!}
Finally, we use \verb|\begin{document}| to tell \LaTeX{} that what follows
is the contents we actually care to typeset, say hello, and tell \LaTeX{}
we are finished with \verb|\end{document}|.

So far, so good.
Let's cover some basic rules.

\section{Spacing}

\LaTeX{} generally handles inter-word spacing for you, regardless of how many
spaces you type. For example,
\begin{leftfigure}
\begin{lstlisting}
The number  of   spaces    between words doesn't   matter. See?

An empty line starts a new paragraph.
\end{lstlisting}
\end{leftfigure}
yields
\begin{leftfigure}
\lm The number  of   spaces    between words doesn't   matter. See?

An empty line  starts a new paragraph.
\end{leftfigure}
Notice that \LaTeX{} is also automatically following the normal typographic
convention of leaving more space after a period than one does between words.
One quirk to be aware of, though, is that comments ``eat'' all of the leading
space on the subsequent line, such that
\begin{leftfigure}
\begin{lstlisting}
This% weird, right?
is strange
\end{lstlisting}
\end{leftfigure}
gives
\begin{leftfigure}
This% weird, right?
is strange
\end{leftfigure}

\section{Special Characters}
