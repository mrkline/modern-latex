% Build this document with LuaTeX, a modern Unicode-aware LaTeX engine
% that uses system TTF and OTF font files.
% This is needed for the fontspec, microtype, and nolig packages.
%
% We're using KOMA Script to hand-tune footnotes and TOC appearance.
% It should be available in your texlive distribution,
% which is how most distros package LaTeX.
\documentclass[fontsize=10pt, numbers=endperiod, openany]{scrbook}

% Margins: see http://practicaltypography.com/page-margins.html and
% http://practicaltypography.com/line-length.html
% We're aiming for 80-ish characters per line.
\usepackage[paperwidth=6.25in, paperheight=9.25in,
            layoutwidth=6in, layoutheight=9in,
            layouthoffset=0.125in, layoutvoffset=0.125in,
            inner=0.5in,outer=0.75in,top=0.5in,bottom=0.5in,
            includefoot,
            showcrop
           ]{geometry}

% Font specification.
% Use Equity for the body text,
% Roboto for sans serif,
% and mononoki for monospace text.
% Feel free to pick your own fonts or comment these lines out to use TeX's
% traditional Computer Modern.
\usepackage{fontspec}
\usepackage[fleqn]{amsmath}
%\setmainfont[Ligatures=TeX,
%             Numbers=Lowercase,
%             SmallCapsFont={Equity Caps A},
%             SmallCapsFeatures={StylisticSet=10}, % Set everything in \textsc{} as small caps
%             StylisticSet=01, % Slightly smaller quotes, asterisks, etc.
%            ]{Equity Text A}

% Help, I've fallen down a rabbit hole.
% Define better kerns following f
\directlua
{
  fonts.handlers.otf.addfeature {
    name = "morekerns",
    type = "kern",
    data = {
      ["f"] = {
        ["!"] = 150,
        ["T"] = 120
      },
    },
  }
}

\setmainfont[Ligatures=TeX, Numbers={Proportional,Lowercase},
             BoldFont=AGaramond Pro Semibold,
             BoldItalicFont=AGaramond Pro Semibold Italic,
             RawFeature=+morekerns
            ]{Adobe Garamond Pro}
\setsansfont[Ligatures=TeX,
             Scale=MatchUppercase,
             Style=Alternate, % Straight-legged R
             UprightFont = *-55Rg,
             ItalicFont = *-56It,
             BoldFont = *-65Md,
             BoldItalicFont = *-66MdIt
             ]{NHaasGroteskTXPro}
\setmonofont[Scale=MatchLowercase]{mononoki}

% We'll be using this quite a bit:
\newfontfamily{\lm}[%
    Ligatures=TeX,
    SmallCapsFont = * Caps
]{Latin Modern Roman}

\usepackage{polyglossia}
\setdefaultlanguage[variant=american]{english}
\setotherlanguage{vietnamese}

\usepackage{blindtext}

\usepackage{microtype} % Font expansion, protrusion, and other goodness

% Disable ligatures across grapheme boundaries
% (see the package manual for details.)
\usepackage[american]{selnolig}

% Use symbols for footnotes, resetting each page
\usepackage[perpage,bottom,symbol*]{footmisc}

% Left flush footnotes. See the KOMA Script manual.
\deffootnote[1em]{1em}{1em}{\thefootnotemark}
% Set the width of the rule separating body text and footnotes
\setfootnoterule{0.7\textwidth}

% Like many fonts, Equity's asterisk is already set in a "superscripted" form.
% Superscripting *that* makes it annoyingly small.
% To fix this, we have to redefine footnote marks so that they aren't superscript,
% then raise all the other symbols.
%
% Feel free to remove this if your body type doesn't have this peculiarity,
% but unfortunately many do.
% See http://tex.stackexchange.com/a/16241
%
% We use the Unicode symbols themselves (instead of \dagger, \ddagger, \P, etc.)
% because the latter fall back to Computer Modern/Latin Modern in some cases,
% (e.g., if you're using mathastext instead of unicode-math).
% Alternatively, you could use \textdagger, \textddagger, etc.,
% but this seems more concise.
\DefineFNsymbols*{tweaked}{%
    {*}%
    {\textsuperscript†}%
    {\textsuperscript‡}%
    {\textsuperscript{◊}}%
    {\textsuperscript{¶}}%
    {**}%
    {\textsuperscript{††}}%
    {\textsuperscript{‡‡}}%
}
\setfnsymbol{tweaked}
\deffootnotemark{\thefootnotemark}

\DeclareTOCStyleEntry[%
    beforeskip=5pt,
    entrynumberformat = \addfontfeature{Numbers={Tabular,Uppercase}},
    pagenumberformat = \addfontfeature{Numbers={Tabular,Uppercase}},
    linefill = \TOCLineLeaderFill
]{tocline}{chapter}

% Don't use a sans font for description labels.
\addtokomafont{descriptionlabel}{\rmfamily}
\setkomafont{disposition}{\rmfamily}
\addtokomafont{chapter}{\addfontfeature{Numbers=Uppercase}}
\setkomafont{section}{\Large\itshape}
\renewcommand*\thesection{\upshape\arabic{section}}

%\setcapwidth[l]{.8\textwidth}
%\setcapmargin{0pt}
\setkomafont{caption}{\sffamily\footnotesize}
\setkomafont{captionlabel}{\sffamily\footnotesize}
\renewcommand*{\figureformat}{}
\renewcommand*{\tableformat}{}
\renewcommand*{\captionformat}{}

% Use uppercase numbers for numbered lists.
% (We're using lowercase ones for the body text.)
% See http://tex.stackexchange.com/a/133186
\usepackage{enumitem}
\setlist[enumerate]{font=\addfontfeatures{Numbers=Uppercase}}

% Custom footer
\usepackage{scrlayer-scrpage}
\clearpairofpagestyles
\pagestyle{scrheadings}
\setkomafont{pagefoot}{\upshape}
\cefoot*{\thepage} % \, is a TeX primitive for a thin space.
\cofoot*{\thepage} % \, is a TeX primitive for a thin space.

\usepackage{graphicx}

\newcommand{\codesize}{\fontsize{10pt}{12pt}}

\usepackage{changepage} % For adjustwidth

\usepackage{mflogo} % for METAFONT
\usepackage{metalogo} % for \LuaLaTeX

\usepackage{endnotes}
% Endnotes _mostly_ works... with Koma Script, no less.
% I suppose I can only grumble a little about the hackery below.
\renewcommand\enoteheading{\chapter{Notes}}
\renewcommand\enoteformat{\leavevmode\llap{\makeenmark}}
% OTF goodness...
\renewcommand\makeenmark{{\addfontfeature{VerticalPosition=Superior}\theenmark}}

\usepackage{listings}
\lstset
{
    language=[LaTeX]TeX,
    breaklines=false,
    basicstyle=\ttfamily\small,
    keywordstyle=\ttfamily\small,
}

\newenvironment{leftfigure}
  {\par\vspace{0.5\baselineskip minus 0.25\baselineskip}\begin{adjustwidth}{3em}{0pt}}
  {\end{adjustwidth}\vspace*{0.3\baselineskip minus 0.2\baselineskip}}

\title{Modern LaTeX}
\author{Matt Kline}
\date{\today}

% Custom footer
% Hyperlinks
\usepackage[unicode,pdfusetitle,hidelinks]{hyperref}

% Use \punckern to overlap periods, commas, and footnote markers
% for a tighter look.
% Care should be taken to not make it too tight - f" and the like can overlap
% if you're not careful.
\newcommand{\punckern}{\kern-0.3ex}
% For placing commas close to, or under, quotes they follow.
% We're programmers, and we blatantly disregard American typographical norms
% to put the quotes inside, but we can at least make it look a bit nicer.
\newcommand{\quotekern}{\kern-0.5ex}


% Create an unbreakable string of text in a monospaced font.
% Useful for `command --line --args`
\newcommand{\monobox}[1]{\mbox{\texttt{#1}}}

% C++ looks nicer if the ++ is in a monospace font and raised a bit.
% Also, use uppercase numbers to match the capital C.
\newcommand{\plusplus}{\raisebox{0.2ex}{++}}
\newcommand{\cpp}[1]{C\kern-0.1ex\plusplus{\addfontfeature{Numbers=LowercaseOff}#1}}
\newcommand{\clang}[1]{C{\addfontfeature{Numbers=LowercaseOff}#1}}
\newcommand{\csharp}{C\raisebox{0.25ex}{\#}}

\newcommand{\fig}[1]{Figure~\ref{#1}}

% Italicize new terms
\newcommand{\introduce}[1]{\textit{#1}}

% Letterspace acronyms a bit.
\newcommand{\acronym}[1]{\textsc{\addfontfeature{LetterSpace=8}#1}}

% See http://tex.stackexchange.com/a/68310
\makeatletter
\let\runauthor\@author
\let\rundate\@date
\let\runtitle\@title
\makeatother

% Spend a bit more time to get better word spacing.
% See http://tex.stackexchange.com/a/52855/92465
\emergencystretch=1em

\begin{document}
\fontsize{10pt}{13pt}\selectfont

\frontmatter
\setcounter{secnumdepth}{0}
\setlength\parindent{0pt}

% Custom title instead of \maketitle
\pagenumbering{gobble}
\vspace*{1.5in}
\begin{center}
\fontsize{0.5in}{0.7in}\selectfont
Modern

\fontsize{1in}{1.1in}\selectfont
\LaTeX
\vfill
\LARGE
Matt Kline
\end{center}
\clearpage

\null
\vfill
Some crappy draft, typeset \today.
\vspace*{0.5in}

The author apologizes for typos, misattributions, or any other flubs,
and welcomes you to point them out on this book's Github
repository at \\
\url{https://github.com/mrkline/latex-book}, or via email to \\
\texttt{matt <at> bitbashing.io}

The author does not have a checking account with the Bank of San Serriffe,
but will happily purchase a beverage of your choice the next time we meet.

\vspace*{0.5in}
{\addfontfeature{Numbers={Proportional,Uppercase}}
Copyright © 2018 \\
by \runauthor
\bigskip

This book is licensed under the \\
Creative Commons Attribution-ShareAlike~4.0 International License. \\
The full text of the license is available at \\
\url{https://creativecommons.org/licenses/by-sa/4.0/legalcode}
}
\clearpage

\vspace*{1in}
{\itshape%
%\noindent To Donald, Leslie, \\
%Ellen, Erik, Jost, Matthew, \& Robert
%\bigskip

To Max, who once told me about a cool program he was using to type up
his college papers.
}
\cleardoublepage

\pagenumbering{roman}
\tableofcontents

\mainmatter
\setlength\parindent{1.5em}

\pagenumbering{arabic}
\setcounter{page}{1} % Restart page numbering after the ToC.
\cleardoublepage

\chapter{Typography and You}
\label{typography}

Life is a parade of written language.
Wherever you go,
ads, apps, articles, essays, emails, and messages
shove text in your face.
But when you read that text, you see so much more than the author's
verbiage.
Consciously or not, you notice the shapes and sizes of letters.
You notice how those letters are arranged into words,
how those words are arranged into paragraphs,
how those paragraphs are arranged onto pages and screens.
You notice \introduce{typography}.
% Smoke test: a line should be 2-3 alphabets wide
%\\ abcdefghijklmnopqrstuvwxyzabcdefghijklmnopqrstuvwxyzabcdefghijklmnopqrstuvwxyz
\begin{leftfigure}
\fontspec{TeX Gyre Termes}\fontsize{12bp}{24bp}\selectfont\raggedright
Typography is why these lines evoke memories of awful essays
you wrote in school.
Do many books look this way? Why not?
\end{leftfigure}
\medskip
\noindent There is a reason street signs don't look like this:
\begin{leftfigure}
\fontspec[Scale=MatchUppercase]{KJV1611}\Large E Gorham St.
\end{leftfigure}
And why an important switch in a spaceship might be labeled like this:
\begin{leftfigure}
\fontspec[Scale=MatchUppercase]{Futura-Med}CM/SM SEP
\end{leftfigure}
Not like this:
\begin{leftfigure}
\fontspec{Old Script}\Large CM/SM Sep
\end{leftfigure}

Written communication is as much about the shapes and layout of your words
as it is about the ones you choose.
Good typography isn't just art---it's function.
And if you care about any of this,
you you should try \LaTeX,\punckern\footnote{Pronounced ``lay-tech''
or ``lah-tech''}
a program for crafting written documents.
By carefully arranging subtle details,
it produces beautiful typography with little effort.
Modern versions can also leverage recent\footnote{By recent,
I mean ``from the mid-1990s''\quotekern, but web browsers and desktop publishing
software are only just starting to catch up.} advances in digital typesetting,
offering you the same tools used by professional graphic designers and
publishers.

\section{\texorpdfstring{\LaTeX}{LaTeX}?}

\LaTeX{} is an alternative to ``word processors'' like
Microsoft Word, Apple Pages, Google Docs,
and LibreOffice Writer.
These other applications operate on the principle of
\introduce{What You See Is What You Get}
\acronym{(wysiwyg)}, where what's on screen is the same
as what comes out of your printer.
\LaTeX{} is different. Here, documents are written as
``plain'' text files, using \introduce{markup} to specify
how the final result should look.
If you've done any web development, this is a similar
process---just as \acronym{html} and \acronym{css} describe
the page you want browsers to draw, your markup describes
the appearance of your document to \LaTeX.

\begin{samepage}
\begin{leftfigure}
\begin{lstlisting}
\LaTeX{} is an alternative to ``word processors'' like
Microsoft Word, Apple Pages, Google Docs,
and LibreOffice Writer.
These other applications operate on the principle of
\introduce{What You See Is What You Get}
\acronym{(wysiwyg)}, where what's on screen is the same
as what comes out of your printer.
\LaTeX{} is different. Here, documents are written as
``plain'' text files, using \introduce{markup} to specify
how the final result should look.
If you've done any web development, this is a similar
process---just as \acronym{html} and \acronym{css} describe
the page you want browsers to draw, your markup describes
the appearance of your document to \LaTeX.
\end{lstlisting}
\captionof{figure}{The \LaTeX{} markup for the paragraph above}
\end{leftfigure}
\end{samepage}

This might seem strange if you haven't worked with markup before,
but it comes with a few advantages:
\begin{enumerate}
\item You can handle your writing's content and its presentation separately.
    At the start of each document,
    you describe the design you want.
    \LaTeX{} takes it from there, consistently formatting your whole text.
    Compare this to a \acronym{wysiwyg} system,
    where you constantly deal with appearances
    as you write.
    If you changed the look of a caption,
    were you sure to find all the other captions and do the
    same?
    If the program formats something in a way you don't like,
    is it hard to fix?%\footnote{I spent far too much of my childhood
    %fighting with Word about how it wrapped text around images.}

\item You can define your own commands, then tweak them to instantly adjust
    every place they're used.
    For example, the \verb|\introduce| and \verb|\acronym| commands
    from the example paragraph are my own creations.
    The former \introduce{italicizes} text, and the latter sets words in
    \acronym{small caps} with a bit of extra
    \mbox{\textsc{\addfontfeature{LetterSpace=15}letterspacing}} so the characters
    don't look \textsc{\addfontfeature{LetterSpace=-10}too crowded}.
    If I decide tomorrow that I would rather introduce new terms
    \textbf{\itshape with this look}, or that acronyms should look
    {\small\addfontfeature{LetterSpace=6} LIKE THIS},
    I just change the two lines that define those commands.
    Every spot in this book that uses them is immediately updated.

\item Being able to save the document as plain text also has benefits:
    \begin{itemize}
    \item It can be read and understood with any text editor.
    \item Structure is immediately visible
        and simple to replicate.\punckern\footnote{Compare this to
        \acronym{wysiwyg} systems, where it's not always obvious
        how certain formatting was produced or how to replicate it.}
    \item Content is easily generated by scripts and programs.
    \item Changes can be tracked with standard version control software,
        like Git or Mercurial.
    \end{itemize}
\end{enumerate}

\section{Another guide?}

You might wonder why the world needs another guide for \LaTeX{}.
After all, it's existed for decades.
A quick Amazon search finds nearly a dozen books on the topic.
There are plenty of great resources online.

Unfortunately, most of the introductions you will find have two fatal flaws:
they are long, and they are old.
A 200+ page book seems daunting to anybody just trying to learn the basics,
and age matters because of how much typesetting has changed since 1986.
When \LaTeX{} was first released that year, none of the publishing technologies
we use today existed.
Adobe wouldn't debut their Portable Document Format for seven more years,
and desktop publishing was a fledgling curiosity.
This shows---badly---in most \LaTeX{} guides.
If you look for instructions to change your document's font,
you'll get swamped with bespoke nonsense.\punckern\footnote{%
Take any criticisms that you find here with a grain of
salt. After all, the fact that all of the technology around \LaTeX{} became
obsolete---multiple times---is a testament to its staying power.}

The good news is that  \LaTeX{} has improved by leaps and bounds in recent years.
It's time for a guide that doesn't weigh you down with decades of legacy
or try (in vain) to be a comprehensive reference.
After all, you're a smart, resourceful individual who sling a search engine.
This book will:

\begin{enumerate}
\item Teach you the fundamentals of \LaTeX.
\item Point you to places where you can learn more.
\item Show you how to use modern typesetting technologies and techniques,
    like OpenType and microtypography.
\item End promptly thereafter.
\end{enumerate}
\vspace{\baselineskip}

\noindent Let's begin.


\chapter{Installation}
\label{installation}

You install \LaTeX{} on your computer as a \introduce{distribution}.
It comes with:
\begin{enumerate}
\item \LaTeX, the program---the thing that turns text files into
    documents.\footnote{Well, actually, multiple \LaTeX{} programs,
    but we're getting to that.}
\item A common set of \LaTeX{} \introduce{packages}.
    Packages are bundles of code that do all sorts of things,
    like provide new commands or change a document's style.
    We'll see lots of them in action throughout this book.
\item Miscellaneous tools, like editors.
\end{enumerate}
Each major operating system has its own \LaTeX{} distribution:
\begin{description}
\item[Mac OS] has Mac\TeX. Grab it from \http{www.tug.org/mactex}
    and install it using the instructions there.

\item[Windows] has Mik\TeX.
    Install it from \https{miktex.org/download}.
    Mik\TeX{} has the helpful ability to automatically download
    additional packages as your documents use them for the first time.

\item[Linux and BSD] use \TeX{} Live.
    Like most software, it is provided through your
    \acronym{os}'s package manager.
    Linux distributions usually contain a \texttt{texlive-\allowbreak full}
    or \texttt{texlive-\allowbreak most} package that installs everything
    you need.\punckern\footnote{%
    If you'd rather keep the install size down,
    Linux distributions usually break \TeX{} Live into multiple distro packages.
    Look for ones with names like
    \texttt{texlive-\allowbreak core}, \texttt{texlive-\allowbreak luatex}
    and \texttt{texlive-\allowbreak xetex}.
    As you work with \LaTeX, you may need less-common packages,
    which usually have names like \texttt{texlive-\allowbreak latexextra},
    \texttt{texlive-\allowbreak science}, and so on.
    Of course, all of this may vary from one Linux distribution to another.}
\end{description}

\section{Editors}

Since \LaTeX{} source files are regular text files,
you write them with the usual choices: Vim, Emacs,
Sublime, VS~Code, and so on.\punckern\footnote{If you've never used
any of these, try a few.
They're popular with programmers and other folks who shuffle text around
screens all day. Just don't use Notepad. Life is too short.}
There are also editors designed specifically for \LaTeX{},
which often come with their own built-in \acronym{pdf} viewer.
(You can find a fairly comprehensive list on the \LaTeX{} Wikibook,
in its installation chapter. See Appendix~\ref{resources}.)

\section{Online options}

If you can't be bothered to install \LaTeX{} on your computer,
try online editors like Share\LaTeX{} or Overleaf.
This book won't focus on these web-based tools,
but the same basics apply.
Of course, you have less control over certain aspects
like available fonts, the version of \LaTeX{} that's used, and so on.


\chapter{Hello, \texorpdfstring{\LaTeX}{LaTeX}!}

Now that you have a \LaTeX{} distribution installed,
let's try it out.
Open up your editor of choice and save the following as \texttt{hello.tex}:
\begin{leftfigure}
\begin{lstlisting}
\documentclass{article}
% Say hello
\begin{document}
Hello, World!
\end{document}
\end{lstlisting}
\end{leftfigure}
Next, we'll run this file through \LaTeX{} (the program)\footnote{Not to be
confused with \LaTeX{} the lunchbox, \LaTeX{} the breakfast cereal,
or \LaTeX{} the flamethrower. The kids love this stuff!}
to get our document.
The installation placed several different versions---or
\introduce{engines}---on your machine,
but for this entire book, we'll always use \LuaLaTeX{} or \XeLaTeX.
These are the newest engines available---see Appendix~\ref{history} for an
explanation of how they differ from the others.

If you are using a \LaTeX{}-specific editor, it probably contains a drop-down
menu or some other configuration to select the engine you'd like to use,
as well as a button to generate your document.
Otherwise, from a terminal,\punckern\footnote{How to work a terminal emulator,
making sure the newly-installed \LaTeX{} programs are in your \texttt{PATH},
and so on are all outside the scope of this book.
As is tradition, the leading dollar sign in examples just denotes a console
prompt, and shouldn't actually be typed.}
run the following:
\begin{leftfigure}
\begin{lstlisting}
$ xelatex hello.tex
\end{lstlisting}
\end{leftfigure}
Feel free to use \texttt{lualatex} instead---there are a few differences
between the two that we'll discuss later, but either is fine for now.
With luck, you should see some output that ends in a message like:
\begin{leftfigure}
\begin{lstlisting}
Output written on hello.pdf (1 page).
Transcript written on hello.log.
\end{lstlisting}
\end{leftfigure}
And in your current directory, you should find a newly minted \texttt{hello.pdf}.
Open it up and you should see a page with this at the top:
\begin{leftfigure}
\lm Hello, World!
\end{leftfigure}
Congrats,
you just created your first document!
Let's unpack what we did.

All \LaTeX{} documents begin with a \verb|\documentclass| declaration,
which picks a base ``style'' to use.
Many classes are available---and you can even create your own---but common
ones include \texttt{article}, \texttt{report}, \texttt{book},
and \texttt{beamer}.\punckern\footnote{This last one is for slideshows.
Kind of an odd name, no?}
For your average document, \texttt{article} is probably a good choice.
The next line, \verb|% Say hello|,
is a \introduce{comment}---anything placed after a percent symbol on a line
is ignored by the engine,
so we can use \texttt{\%} to leave notes for anybody reading
the document's source.\punckern\footnote{Including, perhaps most importantly,
a confused version of your future self!}
Finally, we use \verb|\begin{document}| to tell \LaTeX{} that what follows
is our actual contents,
and we use \verb|\end{document}| to state that we are finished.

Let's cover some more basics.

\section{Spacing}

\LaTeX{} generally handles inter-word spacing for you, regardless of how many
times you hit the space bar.\punckern\footnote{Or the tab key}
For example,
\begin{leftfigure}
\begin{lstlisting}
The number  of   spaces    between words doesn't   matter. The same
is true for sentences.

An empty line starts a new paragraph.
\end{lstlisting}
\end{leftfigure}
yields
\begin{leftfigure}
\lm The number  of   spaces    between words doesn't   matter. The same
is true for sentences.

An empty line  starts a new paragraph.
\end{leftfigure}
Notice that \LaTeX{} automatically follows typographic
conventions, such as indenting new paragraphs and leaving more space after a
period than it leaves between words.
One quirk to be aware of is that comments ``eat'' all of the leading
space on the subsequent line, such that
\begin{leftfigure}
\begin{lstlisting}
This% weird, right?
  is strange
\end{lstlisting}
\end{leftfigure}
gives
\begin{leftfigure}
This% weird, right?
  is strange
\end{leftfigure}

\section{Commands}

\LaTeX{} provides various commands to issue instructions to
the engine, and you can define your own as well.
Their names always begin with a backslash (\,\texttt{\textbackslash}\,),
contain only letters, and are case-sensitive.\punckern\footnote{%
% \verb doesn't play nicely in footnotes, so...
\texttt{\textbackslash foo}
is different from \texttt{\textbackslash Foo}, for example.}
Some commands require parameters, or \introduce{arguments}---\verb|\documentclass|,
for example, needs to know which class we want.
Arguments are enclosed in subsequent pairs of braces,
so if some command needed two arguments, we would type:
\begin{leftfigure}
\begin{lstlisting}
\somecommand{argument1}{argument2}
\end{lstlisting}
\end{leftfigure}
Many commands also take optional arguments, which are enclosed in square
brackets and precede the mandatory ones.
Say you want to inform \LaTeX{} that your document will be printed as
double-sided pages.\punckern\footnote{\texttt{twoside} introduces commands
that only make sense in the context of double-sided printing,
such as ones that skip to the start of the next odd page.
It also allows you to have different margins for even and odd pages,
which is useful for texts like this book.}
This is done with an optional argument to \verb|\documentclass|:
\begin{leftfigure}
\begin{lstlisting}
\documentclass[twoside]{article}
\end{lstlisting}
\end{leftfigure}

Other commands take no arguments at all, such as \verb|\LaTeX|,
which prints the \LaTeX{} logo.
Know that these commands consume any space that follows them.
For example,
\begin{leftfigure}
\begin{lstlisting}
\LaTeX is great, but it can be strange sometimes.
\end{lstlisting}
\end{leftfigure}
will give you
\begin{leftfigure}
\lm \LaTeX is great, but it can be strange sometimes.
\end{leftfigure}
You can fix this by adding an empty pair of braces to the command.
Of course, the braces aren't needed if there is no space to preserve:
\begin{leftfigure}
\begin{lstlisting}
Let's learn \LaTeX! \LaTeX{} is a powerful tool,
but a few of its rules are a little weird.
\end{lstlisting}
\end{leftfigure}
gets us
\begin{leftfigure}
\lm Let's learn \LaTeX! \LaTeX{} is a powerful tool,
but a few of its rules are a little weird.
\end{leftfigure}

\section{Special characters and line breaks}

Some characters have special meanings in \LaTeX.
We saw above, for example, that \verb|%| starts a comment
and \verb|\| starts a command.
The full list of special characters is:
\begin{leftfigure}
\begin{lstlisting}
# $ % ^ & _ { } ~ \
\end{lstlisting}
\end{leftfigure}
Each has a corresponding command
to print it in your document:
\begin{leftfigure}
\begin{lstlisting}
\# \$ \% \^{} \& \_ \{ \} \~{} \textbackslash
\end{lstlisting}
\end{leftfigure}
will produce
\begin{leftfigure}
\lm \# \$ \% \^{} \& \_ \{ \} \~{} \textbackslash
\end{leftfigure}
Regardless of what comes after them, you must always add braces to
the caret (\,\texttt{\^{}}\,) and tilde (\,\~{}\,).
This is a relic from days when these commands were also used to produce
\introduce{diacritical marks}:
once upon a time, users had to typeset ``jalapeño'' as
\verb|jalape\~no|.
Today, we just type \texttt{ñ} into our source
file.\punckern\footnote{The ease with which you can do this depends
on the keyboard you're using, the language settings in your \acronym{os},
and your editor.
Later on, we'll talk much more about non-English languages and Unicode fun.}

If you're wondering why we print \texttt{\textbackslash} with
\verb|\textbackslash| instead of just \verb|\\|,
it's because the latter is the command to force a line break.
\begin{leftfigure}
\begin{lstlisting}
Give me \\
a brand new line!
\end{lstlisting}
\end{leftfigure}
obeys:
\begin{leftfigure}
\lm Give me \\
a brand new line!
\end{leftfigure}
Use this power judiciously---deciding how to best break paragraphs into lines
is one of \LaTeX{}'s greatest skills.

\section{Environments}

We often format text in \LaTeX{} by placing it into \introduce{environments}.
These always start with \verb|\begin{name}| and conclude with \verb|\end{name}|,
where \texttt{name} is that of the desired environment.
Take \texttt{quote}, which adds additional spacing on both sides of a block
quotation:
\begin{leftfigure}
\begin{lstlisting}
Donald Knuth once wrote,
\begin{quote}
We should forget about small efficiencies,
say about 97\% of the time:
premature optimization is the root of all evil.
Yet we should not pass up our opportunities in that critical 3\%.
\end{quote}
\end{lstlisting}
\end{leftfigure}
quotes
\begin{leftfigure}
\lm
Donald Knuth once wrote,
\begin{quote}
We should forget about small efficiencies,
say about 97\% of the time:
premature optimization is the root of all evil.
Yet we should not pass up our opportunities in that critical 3\%.
\end{quote}
\end{leftfigure}

\section{Groups and command scope}
Some commands change how \LaTeX{} typesets the following text.
\verb|\itshape|, for example, \textit{italicizes} everything that comes after it.
To limit a command's effect to a certain region, we surround with braces.
\begin{leftfigure}
\begin{lstlisting}
{\itshape Sometimes we want italics}, but only sometimes.
\end{lstlisting}
\end{leftfigure}
is set as
\begin{leftfigure}
\lm {\itshape Sometimes we want italics}, but only sometimes.
\end{leftfigure}
The text within the braces forms a \introduce{group},
and all commands issued inside the group only take effect until it ends.
Environments also form implicit groups:
\begin{leftfigure}
\begin{lstlisting}
\begin{quote}
\itshape If I italicize a quote, the following text will
use upright type again.
\end{quote}
See? Back to normal.
\end{lstlisting}
\end{leftfigure}
produces
\begin{leftfigure}
\lm
\begin{quote}
\itshape If I italicize a quote, the following text will
use upright type again.
\end{quote}
See? Back to normal.
\end{leftfigure}
One other use of groups is to get around the spacing oddities of zero-argument
commands: some prefer \verb|{\LaTeX}| over \verb|\LaTeX{}|.


\appendix

\chapter{A Brief History of \texorpdfstring{\LaTeX}{LaTeX}}

\label{history}

Donald Knuth is celebrated among programmers as
the man who coined the term \emph{analysis of algorithms} in 1968
and pioneered many of the computer science fundamentals we use today.
Knuth is perhaps most famous for his ongoing magnum opus,
\textit{The Art of Computer Programming}.

When the first volume of \acronym{taocp} was released that same year,
it was printed the way most books had been since the turn of the century:
with \introduce{hot metal} type.
Each individual letter was cast from molten lead,
then arranged into its line.
These lines were clamped together to form pages,
which were finally inked and pressed against paper.

By March of 1977, Knuth was ready for a second run of \acronym{taocp}, volume~2,
but he was horrified when he received the proofs.
Hot metal typesetting was an expensive, complicated, and time-consuming process,
so publishers had replaced it with phototypesetting,
which works by projecting images of characters onto film.
The new technology, while much cheaper and faster,
didn't provide the quality Knuth
expected.\punckern\endnote{Knuth, \textit{Digital Typography} (Stanford, 1999), 3--5}

The average author would have resigned themselves to this change and moved on,
but Knuth took great pride in his books' appearances,
especially their mathematics.
Around this time, he also discovered the growing field of digital typesetting,
where glyphs are built from tiny dots,
packed together at over 1,000 per inch.
Inspired,
Knuth set off on one of the greatest yak shaves\footnote{Programmers
call seemingly unrelated work needed to solve their main problem
``yak shaving''\quotekern. The phrase is thought to originate from an episode
of \textit{The Ren~\&~Stimpy Show}.\punckern\endnote{``yak shaving''\quotekern,
\textit{The Jargon File},
\href{http://www.catb.org/~esr/jargon/html/Y/yak-shaving.html}%
{\texttt{www.catb.org/\~{}esr/jargon/html/Y/yak-shaving.html}}}}
of all time.
For years, he paused work on his books to create his own
typesetting system.
When the dust settled in 1978, Knuth had the first version of
\TeX.\punckern\footnote{The name ``\TeX{}'' comes from the Greek
{\fontspec[Scale=MatchLowercase]{NotoSerif-Medium}τέχνη},
meaning \introduce{art} or \introduce{craft}.\punckern\endnote{Knuth,
\textit{The \TeX book}, 1}}

It's hard to appreciate how much of a revolution \TeX{} was,
especially looking back from a time where anybody with a copy
of Word can be their own desktop publisher.
Adobe's \acronym{pdf} wouldn't exist for another decade, so Knuth
and his graduate students devised their own device-independent format,
\acronym{dvi}.
Scalable fonts were uncommon, so he created \MF{} to rasterize his glyphs
into dots on the page.
Perhaps most importantly, Knuth and his students designed algorithms
to automatically hyphenate and justify text into
beautifully-typeset paragraphs.\punckern\footnote{These same algorithms went
on to influence the ones Adobe uses in its software today.\punckern\endnote{%
Several sources (\http{www.tug.org/whatis.html},
\https{tug.org/interviews/thanh.html},
\http{www.typophile.com/node/34620})
mention \TeX's influence on the \textit{hz}-program by Peter Karow
and Hermann Zapf, thanks to via Knuth's collaborations with Zapf.
\textit{hz} was later acquired by Adobe and used
when creating InDesign's paragraph formatting systems.}}

\LaTeX{}, short for Lamport~\TeX{}, was later developed by Leslie Lamport
as a set of commands for common document layouts.
It was introduced in 1986 with his guide,
\textit{\LaTeX: A~Document Preparation System}.
Other typesetting systems based on \TeX{} also exist,
the other most popular today being Con\TeX{}t.

Development continues,
both in the form of user-provided packages for \TeX{} and \LaTeX{},
and on improvements to the \TeX{} typesetting program itself.
Currently, there are four versions, or \introduce{engines}:
\begin{description}
\item[\TeX] is the original system by Donald Knuth.
Knuth stopped adding features after version 3.0 in March~1990,
and all subsequent releases have contained only bug fixes.
With each release, the version number asymptotically approaches $\pi$
by adding an additional digit.
The most recent version, 3.14159265, came out in January~2014.

\item[pdf\TeX] is an extension of \TeX{} that provides direct \acronym{pdf}
    output (instead of \TeX's \acronym{dvi}),
    native support for PostScript\footnote{%
    \addfontfeature{Numbers=Uppercase}Specifically, PostScript Type 1}
    and TrueType fonts,
    and micro-typographic features discussed in \chapref{microtype}.
    It was originally developed by
    Hàn Thế Thành
    as part of his PhD thesis
    for Masaryk University in Brno, Czech Republic.\punckern\endnote{%
    Hàn Thế Thành,
    \textit{Micro-typographic extensions to the \TeX{} typesetting system}
    (Masaryk University Brno, October 2000)}

\item[\XeTeX] is a further extension of \TeX{} that adds native support for
    Unicode and OpenType.
    It was originally developed by Jonathan Kew in the early 2000s,
    and gained full cross-platform support and inclusion in the \TeX{} Live
    distribution in 2007.\punckern\endnote{Jonathan Kew,
    ``\XeTeX{} Live''\quotekern, \textit{TUGboat} 29, no.~1 (2007)}

\item[\LuaTeX] is similar to \XeTeX{} in its native Unicode and modern font support.
    It also embeds the Lua scripting language into the engine,
    exposing an interface for package and document authors.
    It first appeared in 2007 and is developed by a core team of
    Hans Hagen, Hartmut Henkel, Taco Hoekwater,
    and Luigi Scarso.\punckern\endnote{\http{www.luatex.org}}
\end{description}

Building \TeX{} today is an\dots{} interesting endeavor.
When it was written in the late 1970s,
there were no large, well-documented open-source projects for students to study,
so Knuth set out to make \TeX{} into one.
As part of this effort, \TeX{} was written in a style he calls
\introduce{literate programming}: opposite most programs---where
documentation is interspersed throughout the code---Knuth wrote \TeX{} as a book,
with the code interspersed between paragraphs.
This mix of English and code is called \texttt{WEB}.\punckern\footnote{Knuth
also released a pair of companion programs named
\texttt{TANGLE} and \texttt{WEAVE}.
The former extracts the book---as \TeX, of course---and the latter
produces \TeX's Pascal source code.}

Unsurprisingly, most modern systems don't have good tooling for the late 1970s
dialect of Pascal that \TeX{} was written in,
so present-day distributions use another program,
\texttt{web2c}, to convert its \texttt{WEB} source into C code.
pdf\TeX{} and \XeTeX{} are built by combining the result with other C
and \cpp{} sources.
Instead of following this complicated process,
the \LuaTeX{} authors hand-translated Knuth's Pascal into C
and have been using the resulting code since 2009.\punckern\endnote{%
Taco Hoekwater, \textit{\LuaTeX{} says goodbye to Pascal}
(MAPS 39, Euro\TeX{} 2009),
\https{www.tug.org/TUGboat/tb30-3/tb96hoekwater-pascal.pdf}}



\setlength\parskip{0.8\baselineskip}
\setlength\parindent{0pt}

\chapter{Additional Resources}
\label{resources}

\section{For \texorpdfstring{\LaTeX}{LaTeX}}

As promised at the start, this book is incomplete.
To keep things short,
entire \LaTeX{} features---like figures, captions, references,
and graphics---haven't been mentioned.
Use some of these resources to fill in the gaps,
or as more thorough references:

\begin{adjustwidth}{1.5em}{0pt}
The \LaTeX{} Wikibook, at \url{https://en.wikibooks.org/wiki/LaTeX}

The \TeX{} Stack Exchange, at \url{https://tex.stackexchange.com/}

\textit{The Not So Short Introduction to \LaTeX}, \\
available at \url{https://www.ctan.org/tex-archive/info/lshort/english/}

The Share\LaTeX{} knowledge base, at \url{https://www.sharelatex.com/learn}

\end{adjustwidth}

\section{For typography}

We've spent most of our time here focusing on questions of \emph{what}
you can do with \LaTeX,
and little on \emph{how} you should use it to create quality typography.
Read on:

\begin{adjustwidth}{1.5em}{0pt}
\textit{Practical Typography}, by Matthew Butterick. \\
Available (for free!) at \url{https://practicaltypography.com}

\textit{Stop Stealing Sheep \& Find Out How Type Works}, by Erik Spikermann

\textit{Thinking With Type}, by Ellen Lupton

\textit{The Elements of Typographic Style}, by Robert Bringhurst

\textit{Detail in Typography}, by Jost Hochuli
\end{adjustwidth}

\backmatter

\setkomafont{chapter}{\Huge\itshape}

{\raggedright
\renewcommand\makeenmark{\theenmark.\enspace}
% Chicago Manual of Style, Notes & Bib style, ish.
% http://www.chicagomanualofstyle.org/tools_citationguide/citation-guide-1.html
\theendnotes
}

% Redefine cleardoublepage so the Colophon doesn't demand a front page.
% From https://tex.stackexchange.com/a/24068/92465
{\let\cleardoublepage\clearpage \chapter{Colophon}}

This guide was typeset with \LuaLaTeX{} in Garamond Premier by Robert Slimbach.
His revival is based on Roman type by
\otford{16}{th} century French
punchcutter Claude Garamond.
Italics are inspired by the work of Garamod's contemporary Robert Granjon.

Monospaced items are set in Matthias Tellen's
\href{https://madmalik.github.io/mononoki/}{\texttt{mononoki}},
a typeface designed to work well on both low-resolution computer monitors
and in high-resolution print.

Captions are set in
\href{http://www.fontbureau.com/NHG/}{\textsf{\small Neue Haas Grotesk}},
a Helvetica restoration by Christian Schwartz.
Other digitizations of the classic Swiss typeface are based on fonts made for
Linotype and phototypesetting machines,
resulting in digital versions with all the compromises and kludges from those
past two generations of printing technology.
Schwartz based his work on Helvetica's original drawings,
producing a design faithful to the original cold metal type.

{\fontspec[Ligatures=TeX, Scale=MatchLowercase]{Futura-Boo}URW Futura}
makes a few guest appearances.
Originally released in 1927 by Paul Renner,
Futura has found itself almost everywhere,
from advertising and political campaigns to the moon.
Douglas Thomas's recent history of the typeface,
\textit{Never Use Futura}, is a fantastic read.

Various bits of non-Latin text are set in
{\fontspec[Ligatures=TeX,Scale=MatchLowercase]{NotoSerif-Regular}Noto},
a type family by Google that covers \emph{every} language
in the Unicode standard.

Finally,
{\lm Latin Modern}---the OpenType version of Knuth's Computer Modern used throughout
the book---as well
as {\fontspec[Scale=MatchUppercase]{TeX Gyre Termes}\TeX{} Gyre Termes}---the
free alternative to Times Roman seen on page \pageref{typography}---are from
the digital type foundry of Grupa Użytkowników Systemu \TeX{},
the Polish \TeX{} Users' Group.
An overview of their excellent work can be found at the following locations:\\
\url{http://www.gust.org.pl/projects/e-foundry/latin-modern} \\
\url{http://www.gust.org.pl/projects/e-foundry/tex-gyre}.


\end{document}
