% Build this document with LuaTeX, a modern Unicode-aware LaTeX engine
% that uses system TTF and OTF font files.
% This is needed for the fontspec, microtype, and nolig packages.
%
% We're using KOMA Script to hand-tune footnotes and TOC appearance.
% It should be available in your texlive distribution,
% which is how most distros package LaTeX.
\documentclass[fontsize=11bp, numbers=endperiod, draft=false]{scrbook}

% Margins: see http://practicaltypography.com/page-margins.html and
% http://practicaltypography.com/line-length.html
% We're aiming for 80-ish characters per line.
\usepackage[a5paper,
            inner=0.6in,outer=0.6in,top=0.75in,bottom=0.5in,
            footnotesep=11bp,
            footskip=3em,
            includefoot,
           ]{geometry}

\usepackage[fleqn]{amsmath}

\usepackage{fontspec}

\setmainfont[
    Ligatures=TeX,
    Numbers={Proportional,Lowercase},
    % Coefficients for font-provided space, stretch, and shrink.
    % Garamond Premier is a bit tighter than Adobe Garamond,
    % but stretching it out about 20% gives about the same metrics.
    WordSpace={1.2,1.2,1},
    UprightFeatures={
        SizeFeatures={
            {Size={-10},Font=garamondpremrpro-capt},
            {Size={10-15},Font=garamondpremrpro},
            {Size={15-23},Font=garamondpremrpro-subh},
            {Size={23-},Font=garamondpremrpro-disp}
        },
    },
    BoldFeatures={
        SizeFeatures={
            {Size={-10},Font=garamondpremrpro-smbdcapt},
            {Size={10-15},Font=garamondpremrpro-smbd},
            {Size={15-23},Font=garamondpremrpro-smbdsubh},
            {Size={23-},Font=garamondpremrpro-smbddisp}
        },
    },
    ItalicFeatures={
        SizeFeatures={
            {Size={-10},Font=garamondpremrpro-itcapt},
            {Size={10-15},Font=garamondpremrpro-it},
            {Size={15-23},Font=garamondpremrpro-itsubh},
            {Size={23-},Font=garamondpremrpro-itdisp}
            },
    },
    BoldItalicFeatures={
        SizeFeatures={
            {Size={-10},Font=garamondpremrpro-smbditcapt},
            {Size={10-15},Font=garamondpremrpro-smbdit},
            {Size={15-23},Font=garamondpremrpro-smbditsubh},
            {Size={23-},Font=garamondpremrpro-smbditdisp}
        },
    },
]{Garamond Premier Pro}

\setsansfont[Ligatures=TeX,
             Scale=MatchUppercase,
             Style=Alternate, % Straight-legged R
             UprightFont = *-55Rg,
             ItalicFont = *-56It,
             BoldFont = *-65Md,
             BoldItalicFont = *-66MdIt
             ]{NHaasGroteskTXPro}

\setmonofont[
    Numbers=SlashedZero,
    Scale=MatchLowercase,
    UprightFeatures={
        SizeFeatures={
            {Size={-10}, Font=DriveMono-Book},
            {Size={10-}, Font=DriveMono-Regular},
        },
    },
    ItalicFeatures={
        SizeFeatures={
            {Size={-10}, Font=DriveMono-BookItalic},
            {Size={10-}, Font=DriveMono-Italic},
        },
    },
    BoldFont=DriveMono-Bold,
    BoldItalicFont=DriveMono-BoldItalic
]{Drive Mono}

\setmathrm{Latin Modern Roman}

% We'll be using this quite a bit:
\newfontfamily{\lm}[
    Ligatures=TeX,
    SmallCapsFont = * Caps
]{Latin Modern Roman}
\newfontfamily{\lt}{Latin Modern Mono}

\usepackage{polyglossia}
\setdefaultlanguage[variant=american]{english}
\setotherlanguage{french}

\usepackage{microtype} % Font expansion, protrusion, and other goodness

% Disable ligatures across grapheme boundaries
% (see the package manual for details.)
\usepackage[american]{selnolig}
% I have no idea why Garamond has this on by default.
\nolig{Th}{T|h}

% Use symbols for footnotes, resetting each page
\usepackage[perpage,bottom,symbol*]{footmisc}

% Left flush footnotes. See the KOMA Script manual.
\deffootnote[1em]{1em}{1em}{\thefootnotemark}
% Set the width of the rule separating body text and footnotes
\setfootnoterule{0.8\textwidth}

% Like many fonts, Equity's asterisk is already set in a "superscripted" form.
% Superscripting *that* makes it annoyingly small.
% To fix this, we have to redefine footnote marks so that they aren't superscript,
% then raise all the other symbols.
%
% Feel free to remove this if your body type doesn't have this peculiarity,
% but unfortunately many do.
% See http://tex.stackexchange.com/a/16241
%
% We use the Unicode symbols themselves (instead of \dagger, \ddagger, \P, etc.)
% because the latter fall back to Computer Modern/Latin Modern in some cases,
% (e.g., if you're using mathastext instead of unicode-math).
% Alternatively, you could use \textdagger, \textddagger, etc.,
% but this seems more concise.
\DefineFNsymbols*{tweaked}{%
    {*}%
    {\textsuperscript†}%
    {\textsuperscript‡}%
    {\textsuperscript{◊}}%
    {\textsuperscript{§}}%
    {**}%
    {\textsuperscript{††}}%
    {\textsuperscript{‡‡}}%
}
\setfnsymbol{tweaked}
\deffootnotemark{\thefootnotemark}

\DeclareTOCStyleEntry[%
    beforeskip=8bp,
    entryformat = \bfseries,
    entrynumberformat = \addfontfeature{Numbers={Tabular,Uppercase}},
    pagenumberformat = \addfontfeature{Numbers={Tabular,Uppercase}},
    linefill = \TOCLineLeaderFill
]{tocline}{chapter}
\DeclareTOCStyleEntry[%
    beforeskip=1bp,
    entrynumberformat = \addfontfeature{Numbers={Tabular,Uppercase}},
    pagenumberformat = \addfontfeature{Numbers={Tabular,Uppercase}},
    indent=0.4in
]{tocline}{section}
\DeclareTOCStyleEntry[%
    beforeskip=0pt,
    entrynumberformat = \addfontfeature{Numbers={Tabular,Uppercase}},
    pagenumberformat = \addfontfeature{Numbers={Tabular,Uppercase}},
    indent=0.6in
]{tocline}{subsection}

% Don't use a sans font for description labels.
\addtokomafont{descriptionlabel}{\rmfamily}
% Sections and such use serif type.
\setkomafont{disposition}{\rmfamily}
% Uppercase numbers for chapters.
\addtokomafont{chapter}{\addfontfeature{Numbers=Uppercase}}
% Set size and style of section and subsections
\setkomafont{section}{\Large\itshape}
\setkomafont{subsection}{\large\itshape}

\setcapwidth[c]{.75\textwidth}
%\setcapmargin{0pt}
\setkomafont{caption}{\sffamily\footnotesize}
\setkomafont{captionlabel}{\sffamily\footnotesize}
\renewcommand*{\figureformat}{}
\renewcommand*{\tableformat}{}
\renewcommand*{\captionformat}{}

% Use uppercase numbers for numbered lists.
% (We're using lowercase ones for the body text.)
% See http://tex.stackexchange.com/a/133186
\usepackage{enumitem}
\setlist[enumerate]{font=\addfontfeatures{Numbers=Uppercase}}
\setlist[description]{leftmargin=1em}

\usepackage{tabularx}

% Custom footer
\usepackage[draft=false]{scrlayer-scrpage}
\clearpairofpagestyles
\pagestyle{scrheadings}
\setkomafont{pagefoot}{\upshape}
\cefoot*{\thepage}
\cofoot*{\thepage}

\usepackage{changepage} % For adjustwidth

\usepackage{mflogo} % for METAFONT
\usepackage{metalogo} % for \LuaLaTeX

\usepackage{multicol}

\usepackage{endnotes}
% OTF goodness...
\renewcommand\makeenmark{{\addfontfeature{VerticalPosition=Superior}\theenmark}}

\usepackage{listings}
\lstset
{
    language=[LaTeX]TeX,
    breaklines=false,
    basicstyle=\ttfamily,
    keywordstyle=\ttfamily,
    commentstyle=\ttfamily,
}

% Indent code examples, etc., by double the text size.
\newenvironment{leftfigure}
  {\par\vspace{0.5\baselineskip minus 0.3\baselineskip}\begin{adjustwidth}{22bp}{0pt}}
  {\end{adjustwidth}\vspace{0.5\baselineskip minus 0.3\baselineskip}}

% Like the above, but with no adjustwidth
\newenvironment{flushleftfigure}
  {\par\vspace{0.5\baselineskip minus 0.3\baselineskip}\noindent\ignorespacesafterend}
  {\vspace{0.5\baselineskip minus 0.3\baselineskip}\par\noindent\ignorespacesafterend}

\newenvironment{centerfigure}
  {\par\vspace{0.5\baselineskip minus 0.3\baselineskip}\begin{adjustwidth}{22bp}{22bp}\centering}
  {\end{adjustwidth}\vspace{0.5\baselineskip minus 0.3\baselineskip}}

\usepackage{graphicx}

\usepackage{csquotes}

\title{Modern \texorpdfstring{\LaTeX}{LaTeX}}
\author{Matt Kline}
\date{\today}

% Custom footer
% Hyperlinks
\usepackage[unicode,pdfusetitle,hidelinks]{hyperref}
\hypersetup{
    colorlinks=true, % Use colors
    linkcolor=black, % Intra-doc links are black, like the rest of the text.
    urlcolor=blue % URLs are blue
}

% Use \punckern to overlap periods, commas, and footnote markers
% for a tighter look.
% Care should be taken to not make it too tight - f" and the like can overlap
% if you're not careful.
\newcommand{\punckern}{\kern-0.2ex}
% For placing commas close to, or under, quotes they follow.
% We're programmers, and we blatantly disregard American typographical norms
% to put the quotes inside, but we can at least make it look a bit nicer.
\newcommand{\quotekern}{\kern-0.6ex}


% Create an unbreakable string of text in a monospaced font.
% Useful for `command --line --args`
\newcommand{\monobox}[1]{\mbox{\texttt{#1}}}

\newcommand{\otffrac}[2]{\mbox{%
    {\addfontfeature{VerticalPosition=Superior}#1}%
    ^^^^2044% Unicode fraction slash
    {\addfontfeature{VerticalPosition=Inferior}#2}%
}}
\newcommand{\otford}[2]{\mbox{%
    {\addfontfeature{Numbers=LowercaseOff}#1}%
    {\addfontfeature{VerticalPosition=Superior}#2}%
}}

% C++ looks nicer if the ++ is in a monospace font and raised a bit.
% Also, use uppercase numbers to match the capital C.
\newcommand{\plusplus}{\raisebox{0.1ex}{++}}
\newcommand{\cpp}[1]{C\kern-0.1ex\plusplus{\addfontfeature{Numbers=LowercaseOff}#1}}

% Italicize new terms
\newcommand{\introduce}[1]{\textit{#1}}

% Letterspace acronyms a bit.
\newcommand{\acronym}[1]{\textsc{\addfontfeature{LetterSpace=5}#1}}

% "Chapter <num>" references
\newcommand{\chapref}[1]{chapter~\ref{#1}}

% monospace URLs (without setting the http://...)
\newcommand{\http}[1]{\url{http://#1}}
\newcommand{\https}[1]{\url{https://#1}}

\newcommand{\edition}{Second edition \acronym{(wip)}}

\newcommand{\exercises}{\section{On your own}}

% See http://tex.stackexchange.com/a/68310
\makeatletter
\let\runauthor\@author
\let\rundate\@date
\let\runtitle\@title
\makeatother

% Spend a bit more time to get better word spacing.
% See http://tex.stackexchange.com/a/52855/92465
\emergencystretch=1ex

% Do as I say, not as I do.
\widowpenalty=10000
\clubpenalty=10000

% For an online PDF, blank pages are annoying.
\renewcommand{\cleardoublepage}{\clearpage}

\begin{document}
\fontsize{11bp}{13bp}\selectfont

\frontmatter
\setcounter{secnumdepth}{0}
\setlength\parindent{0pt}

% Custom title instead of \maketitle
\pagenumbering{gobble}
\vspace*{1in}
\begin{center}
\fontsize{0.5in}{0.7in}\selectfont
Modern

\fontsize{1in}{0.9in}\selectfont
\LaTeX

\normalsize
\vspace{1.5\baselineskip}
\edition
\vspace{2in}

\LARGE
\runauthor
\end{center}
\clearpage

{\raggedright%For the page
\null
\vfill
{\addfontfeature{Numbers={Proportional,Uppercase}}
Copyright © 2018--2020 \\
by \runauthor
\bigskip

This book is licensed under the \\
Creative Commons Attribution-ShareAlike~4.0 International License. \\
In short, you are free to share, translate, adapt, or improve this book
so long as you give proper credit and provide your contributions under
the same license. \\
The license's full text is available at \\
\https{creativecommons.org/licenses/by-sa/4.0/legalcode}
\vspace{\baselineskip}

ISBN 978-1-387-80513-6
}

\vspace{\baselineskip}
Print copies of this book will be available at cost from Lulu
once this edition is finalized.
\vfill

The author apologizes for any typos,
f\raisebox{-0.1ex}{o}rmatt\raisebox{0.1ex}{i}ng mistakes,
inaccuracies,
and other flubs.
He welcomes fixes and improvements in this book's Git repository at \\
\https{github.com/mrkline/latex-book}

\vspace{\baselineskip}
Questions, comments, concerns, and diatribes can also be emailed to \\
\texttt{matt <at> bitbashing.io}

\vspace{\baselineskip}
The author does not have a checking account with the Bank of San Serriffe,
but will happily pay for your contributions with a beverage of your choice
next time we meet.

\vspace{0.5in}
\edition{} (online \acronym{pdf}), typeset \today.
} % end ragged right
\clearpage

\vspace*{1in}
{\itshape%
To Max, who once told me about a cool program he used to type up
his college papers.
}
\cleardoublepage

\pagenumbering{roman}
\tableofcontents

\mainmatter
% Indent by one lead, as suggested in The Elements of Typographic Style.
\setlength\parindent{14bp}

\pagenumbering{arabic}
\setcounter{page}{1} % Restart page numbering after the ToC.
\cleardoublepage

\chapter{Typography and You}
\label{typography}

Life is a parade of written language.
Wherever you go,
ads, apps, articles, essays, emails, and messages
shove text in your face.
But when you read that text, you see so much more than the author's
verbiage.
Consciously or not, you notice the shapes and sizes of letters.
You notice how those letters are arranged into words,
how those words are arranged into paragraphs,
how those paragraphs are arranged onto pages and screens.
You notice \introduce{typography}.
% Smoke test: a line should be 2-3 alphabets wide
%\\ abcdefghijklmnopqrstuvwxyzabcdefghijklmnopqrstuvwxyzabcdefghijklmnopqrstuvwxyz
\begin{leftfigure}
\fontspec{TeX Gyre Termes}\fontsize{12bp}{24bp}\selectfont\raggedright
Typography is why these lines evoke memories of awful essays
you wrote in school.
Do many books look this way? Why not?
\end{leftfigure}
\medskip
\noindent There is a reason street signs don't look like this:
\begin{leftfigure}
\fontspec[Scale=MatchUppercase]{KJV1611}\Large E Gorham St.
\end{leftfigure}
And why an important switch in a spaceship might be labeled like this:
\begin{leftfigure}
\fontspec[Scale=MatchUppercase]{Futura-Med}CM/SM SEP
\end{leftfigure}
Not like this:
\begin{leftfigure}
\fontspec{Old Script}\Large CM/SM Sep
\end{leftfigure}

Written communication is as much about the shapes and layout of your words
as it is about the ones you choose.
Good typography isn't just art---it's function.
And if you care about any of this,
you you should try \LaTeX,\punckern\footnote{Pronounced ``lay-tech''
or ``lah-tech''}
a program for crafting written documents.
By carefully arranging subtle details,
it produces beautiful typography with little effort.
Modern versions can also leverage recent\footnote{By recent,
I mean ``from the mid-1990s''\quotekern, but web browsers and desktop publishing
software are only just starting to catch up.} advances in digital typesetting,
offering you the same tools used by professional graphic designers and
publishers.

\section{\texorpdfstring{\LaTeX}{LaTeX}?}

\LaTeX{} is an alternative to ``word processors'' like
Microsoft Word, Apple Pages, Google Docs,
and LibreOffice Writer.
These other applications operate on the principle of
\introduce{What You See Is What You Get}
\acronym{(wysiwyg)}, where what's on screen is the same
as what comes out of your printer.
\LaTeX{} is different. Here, documents are written as
``plain'' text files, using \introduce{markup} to specify
how the final result should look.
If you've done any web development, this is a similar
process---just as \acronym{html} and \acronym{css} describe
the page you want browsers to draw, your markup describes
the appearance of your document to \LaTeX.

\begin{samepage}
\begin{leftfigure}
\begin{lstlisting}
\LaTeX{} is an alternative to ``word processors'' like
Microsoft Word, Apple Pages, Google Docs,
and LibreOffice Writer.
These other applications operate on the principle of
\introduce{What You See Is What You Get}
\acronym{(wysiwyg)}, where what's on screen is the same
as what comes out of your printer.
\LaTeX{} is different. Here, documents are written as
``plain'' text files, using \introduce{markup} to specify
how the final result should look.
If you've done any web development, this is a similar
process---just as \acronym{html} and \acronym{css} describe
the page you want browsers to draw, your markup describes
the appearance of your document to \LaTeX.
\end{lstlisting}
\captionof{figure}{The \LaTeX{} markup for the paragraph above}
\end{leftfigure}
\end{samepage}

This might seem strange if you haven't worked with markup before,
but it comes with a few advantages:
\begin{enumerate}
\item You can handle your writing's content and its presentation separately.
    At the start of each document,
    you describe the design you want.
    \LaTeX{} takes it from there, consistently formatting your whole text.
    Compare this to a \acronym{wysiwyg} system,
    where you constantly deal with appearances
    as you write.
    If you changed the look of a caption,
    were you sure to find all the other captions and do the
    same?
    If the program formats something in a way you don't like,
    is it hard to fix?%\footnote{I spent far too much of my childhood
    %fighting with Word about how it wrapped text around images.}

\item You can define your own commands, then tweak them to instantly adjust
    every place they're used.
    For example, the \verb|\introduce| and \verb|\acronym| commands
    from the example paragraph are my own creations.
    The former \introduce{italicizes} text, and the latter sets words in
    \acronym{small caps} with a bit of extra
    \mbox{\textsc{\addfontfeature{LetterSpace=15}letterspacing}} so the characters
    don't look \textsc{\addfontfeature{LetterSpace=-10}too crowded}.
    If I decide tomorrow that I would rather introduce new terms
    \textbf{\itshape with this look}, or that acronyms should look
    {\small\addfontfeature{LetterSpace=6} LIKE THIS},
    I just change the two lines that define those commands.
    Every spot in this book that uses them is immediately updated.

\item Being able to save the document as plain text also has benefits:
    \begin{itemize}
    \item It can be read and understood with any text editor.
    \item Structure is immediately visible
        and simple to replicate.\punckern\footnote{Compare this to
        \acronym{wysiwyg} systems, where it's not always obvious
        how certain formatting was produced or how to replicate it.}
    \item Content is easily generated by scripts and programs.
    \item Changes can be tracked with standard version control software,
        like Git or Mercurial.
    \end{itemize}
\end{enumerate}

\section{Another guide?}

You might wonder why the world needs another guide for \LaTeX{}.
After all, it's existed for decades.
A quick Amazon search finds nearly a dozen books on the topic.
There are plenty of great resources online.

Unfortunately, most of the introductions you will find have two fatal flaws:
they are long, and they are old.
A 200+ page book seems daunting to anybody just trying to learn the basics,
and age matters because of how much typesetting has changed since 1986.
When \LaTeX{} was first released that year, none of the publishing technologies
we use today existed.
Adobe wouldn't debut their Portable Document Format for seven more years,
and desktop publishing was a fledgling curiosity.
This shows---badly---in most \LaTeX{} guides.
If you look for instructions to change your document's font,
you'll get swamped with bespoke nonsense.\punckern\footnote{%
Take any criticisms that you find here with a grain of
salt. After all, the fact that all of the technology around \LaTeX{} became
obsolete---multiple times---is a testament to its staying power.}

The good news is that  \LaTeX{} has improved by leaps and bounds in recent years.
It's time for a guide that doesn't weigh you down with decades of legacy
or try (in vain) to be a comprehensive reference.
After all, you're a smart, resourceful individual who sling a search engine.
This book will:

\begin{enumerate}
\item Teach you the fundamentals of \LaTeX.
\item Point you to places where you can learn more.
\item Show you how to use modern typesetting technologies and techniques,
    like OpenType and microtypography.
\item End promptly thereafter.
\end{enumerate}
\vspace{\baselineskip}

\noindent Let's begin.

\chapter{Installation}
\label{installation}

You install \LaTeX{} on your computer as a \introduce{distribution}.
It comes with:
\begin{enumerate}
\item \LaTeX, the program---the thing that turns text files into
    documents.\footnote{Well, actually, multiple \LaTeX{} programs,
    but we're getting to that.}
\item A common set of \LaTeX{} \introduce{packages}.
    Packages are bundles of code that do all sorts of things,
    like provide new commands or change a document's style.
    We'll see lots of them in action throughout this book.
\item Miscellaneous tools, like editors.
\end{enumerate}
Each major operating system has its own \LaTeX{} distribution:
\begin{description}
\item[Mac OS] has Mac\TeX. Grab it from \http{www.tug.org/mactex}
    and install it using the instructions there.

\item[Windows] has Mik\TeX.
    Install it from \https{miktex.org/download}.
    Mik\TeX{} has the helpful ability to automatically download
    additional packages as your documents use them for the first time.

\item[Linux and BSD] use \TeX{} Live.
    Like most software, it is provided through your
    \acronym{os}'s package manager.
    Linux distributions usually contain a \texttt{texlive-\allowbreak full}
    or \texttt{texlive-\allowbreak most} package that installs everything
    you need.\punckern\footnote{%
    If you'd rather keep the install size down,
    Linux distributions usually break \TeX{} Live into multiple distro packages.
    Look for ones with names like
    \texttt{texlive-\allowbreak core}, \texttt{texlive-\allowbreak luatex}
    and \texttt{texlive-\allowbreak xetex}.
    As you work with \LaTeX, you may need less-common packages,
    which usually have names like \texttt{texlive-\allowbreak latexextra},
    \texttt{texlive-\allowbreak science}, and so on.
    Of course, all of this may vary from one Linux distribution to another.}
\end{description}

\section{Editors}

Since \LaTeX{} source files are regular text files,
you write them with the usual choices: Vim, Emacs,
Sublime, VS~Code, and so on.\punckern\footnote{If you've never used
any of these, try a few.
They're popular with programmers and other folks who shuffle text around
screens all day. Just don't use Notepad. Life is too short.}
There are also editors designed specifically for \LaTeX{},
which often come with their own built-in \acronym{pdf} viewer.
(You can find a fairly comprehensive list on the \LaTeX{} Wikibook,
in its installation chapter. See Appendix~\ref{resources}.)

\section{Online options}

If you can't be bothered to install \LaTeX{} on your computer,
try online editors like Share\LaTeX{} or Overleaf.
This book won't focus on these web-based tools,
but the same basics apply.
Of course, you have less control over certain aspects
like available fonts, the version of \LaTeX{} that's used, and so on.

\chapter{Hello, \texorpdfstring{\LaTeX}{LaTeX}!}

Now that you have a \LaTeX{} distribution installed,
let's try it out.
Open up your editor of choice and save the following as \texttt{hello.tex}:
\begin{leftfigure}
\begin{lstlisting}
\documentclass{article}
% Say hello
\begin{document}
Hello, World!
\end{document}
\end{lstlisting}
\end{leftfigure}
Next, we'll run this file through \LaTeX{} (the program)\footnote{Not to be
confused with \LaTeX{} the lunchbox, \LaTeX{} the breakfast cereal,
or \LaTeX{} the flamethrower. The kids love this stuff!}
to get our document.
The installation placed several different versions---or
\introduce{engines}---on your machine,
but for this entire book, we'll always use \LuaLaTeX{} or \XeLaTeX.
These are the newest engines available---see Appendix~\ref{history} for an
explanation of how they differ from the others.

If you are using a \LaTeX{}-specific editor, it probably contains a drop-down
menu or some other configuration to select the engine you'd like to use,
as well as a button to generate your document.
Otherwise, from a terminal,\punckern\footnote{How to work a terminal emulator,
making sure the newly-installed \LaTeX{} programs are in your \texttt{PATH},
and so on are all outside the scope of this book.
As is tradition, the leading dollar sign in examples just denotes a console
prompt, and shouldn't actually be typed.}
run the following:
\begin{leftfigure}
\begin{lstlisting}
$ xelatex hello.tex
\end{lstlisting}
\end{leftfigure}
Feel free to use \texttt{lualatex} instead---there are a few differences
between the two that we'll discuss later, but either is fine for now.
With luck, you should see some output that ends in a message like:
\begin{leftfigure}
\begin{lstlisting}
Output written on hello.pdf (1 page).
Transcript written on hello.log.
\end{lstlisting}
\end{leftfigure}
And in your current directory, you should find a newly minted \texttt{hello.pdf}.
Open it up and you should see a page with this at the top:
\begin{leftfigure}
\lm Hello, World!
\end{leftfigure}
Congrats,
you just created your first document!
Let's unpack what we did.

All \LaTeX{} documents begin with a \verb|\documentclass| declaration,
which picks a base ``style'' to use.
Many classes are available---and you can even create your own---but common
ones include \texttt{article}, \texttt{report}, \texttt{book},
and \texttt{beamer}.\punckern\footnote{This last one is for slideshows.
Kind of an odd name, no?}
For your average document, \texttt{article} is probably a good choice.
The next line, \verb|% Say hello|,
is a \introduce{comment}---anything placed after a percent symbol on a line
is ignored by the engine,
so we can use \texttt{\%} to leave notes for anybody reading
the document's source.\punckern\footnote{Including, perhaps most importantly,
a confused version of your future self!}
Finally, we use \verb|\begin{document}| to tell \LaTeX{} that what follows
is our actual contents,
and we use \verb|\end{document}| to state that we are finished.

Let's cover some more basics.

\section{Spacing}

\LaTeX{} generally handles inter-word spacing for you, regardless of how many
times you hit the space bar.\punckern\footnote{Or the tab key}
For example,
\begin{leftfigure}
\begin{lstlisting}
The number  of   spaces    between words doesn't   matter. The same
is true for sentences.

An empty line starts a new paragraph.
\end{lstlisting}
\end{leftfigure}
yields
\begin{leftfigure}
\lm The number  of   spaces    between words doesn't   matter. The same
is true for sentences.

An empty line  starts a new paragraph.
\end{leftfigure}
Notice that \LaTeX{} automatically follows typographic
conventions, such as indenting new paragraphs and leaving more space after a
period than it leaves between words.
One quirk to be aware of is that comments ``eat'' all of the leading
space on the subsequent line, such that
\begin{leftfigure}
\begin{lstlisting}
This% weird, right?
  is strange
\end{lstlisting}
\end{leftfigure}
gives
\begin{leftfigure}
This% weird, right?
  is strange
\end{leftfigure}

\section{Commands}

\LaTeX{} provides various commands to issue instructions to
the engine, and you can define your own as well.
Their names always begin with a backslash (\,\texttt{\textbackslash}\,),
contain only letters, and are case-sensitive.\punckern\footnote{%
% \verb doesn't play nicely in footnotes, so...
\texttt{\textbackslash foo}
is different from \texttt{\textbackslash Foo}, for example.}
Some commands require parameters, or \introduce{arguments}---\verb|\documentclass|,
for example, needs to know which class we want.
Arguments are enclosed in subsequent pairs of braces,
so if some command needed two arguments, we would type:
\begin{leftfigure}
\begin{lstlisting}
\somecommand{argument1}{argument2}
\end{lstlisting}
\end{leftfigure}
Many commands also take optional arguments, which are enclosed in square
brackets and precede the mandatory ones.
Say you want to inform \LaTeX{} that your document will be printed as
double-sided pages.\punckern\footnote{\texttt{twoside} introduces commands
that only make sense in the context of double-sided printing,
such as ones that skip to the start of the next odd page.
It also allows you to have different margins for even and odd pages,
which is useful for texts like this book.}
This is done with an optional argument to \verb|\documentclass|:
\begin{leftfigure}
\begin{lstlisting}
\documentclass[twoside]{article}
\end{lstlisting}
\end{leftfigure}

Other commands take no arguments at all, such as \verb|\LaTeX|,
which prints the \LaTeX{} logo.
Know that these commands consume any space that follows them.
For example,
\begin{leftfigure}
\begin{lstlisting}
\LaTeX is great, but it can be strange sometimes.
\end{lstlisting}
\end{leftfigure}
will give you
\begin{leftfigure}
\lm \LaTeX is great, but it can be strange sometimes.
\end{leftfigure}
You can fix this by adding an empty pair of braces to the command.
Of course, the braces aren't needed if there is no space to preserve:
\begin{leftfigure}
\begin{lstlisting}
Let's learn \LaTeX! \LaTeX{} is a powerful tool,
but a few of its rules are a little weird.
\end{lstlisting}
\end{leftfigure}
gets us
\begin{leftfigure}
\lm Let's learn \LaTeX! \LaTeX{} is a powerful tool,
but a few of its rules are a little weird.
\end{leftfigure}

\section{Special characters and line breaks}

Some characters have special meanings in \LaTeX.
We saw above, for example, that \verb|%| starts a comment
and \verb|\| starts a command.
The full list of special characters is:
\begin{leftfigure}
\begin{lstlisting}
# $ % ^ & _ { } ~ \
\end{lstlisting}
\end{leftfigure}
Each has a corresponding command
to print it in your document:
\begin{leftfigure}
\begin{lstlisting}
\# \$ \% \^{} \& \_ \{ \} \~{} \textbackslash
\end{lstlisting}
\end{leftfigure}
will produce
\begin{leftfigure}
\lm \# \$ \% \^{} \& \_ \{ \} \~{} \textbackslash
\end{leftfigure}
Regardless of what comes after them, you must always add braces to
the caret (\,\texttt{\^{}}\,) and tilde (\,\~{}\,).
This is a relic from days when these commands were also used to produce
\introduce{diacritical marks}:
once upon a time, users had to typeset ``jalapeño'' as
\verb|jalape\~no|.
Today, we just type \texttt{ñ} into our source
file.\punckern\footnote{The ease with which you can do this depends
on the keyboard you're using, the language settings in your \acronym{os},
and your editor.
Later on, we'll talk much more about non-English languages and Unicode fun.}

If you're wondering why we print \texttt{\textbackslash} with
\verb|\textbackslash| instead of just \verb|\\|,
it's because the latter is the command to force a line break.
\begin{leftfigure}
\begin{lstlisting}
Give me \\
a brand new line!
\end{lstlisting}
\end{leftfigure}
obeys:
\begin{leftfigure}
\lm Give me \\
a brand new line!
\end{leftfigure}
Use this power judiciously---deciding how to best break paragraphs into lines
is one of \LaTeX{}'s greatest skills.

\section{Environments}

We often format text in \LaTeX{} by placing it into \introduce{environments}.
These always start with \verb|\begin{name}| and conclude with \verb|\end{name}|,
where \texttt{name} is that of the desired environment.
Take \texttt{quote}, which adds additional spacing on both sides of a block
quotation:
\begin{leftfigure}
\begin{lstlisting}
Donald Knuth once wrote,
\begin{quote}
We should forget about small efficiencies,
say about 97\% of the time:
premature optimization is the root of all evil.
Yet we should not pass up our opportunities in that critical 3\%.
\end{quote}
\end{lstlisting}
\end{leftfigure}
quotes
\begin{leftfigure}
\lm
Donald Knuth once wrote,
\begin{quote}
We should forget about small efficiencies,
say about 97\% of the time:
premature optimization is the root of all evil.
Yet we should not pass up our opportunities in that critical 3\%.
\end{quote}
\end{leftfigure}

\section{Groups and command scope}
Some commands change how \LaTeX{} typesets the following text.
\verb|\itshape|, for example, \textit{italicizes} everything that comes after it.
To limit a command's effect to a certain region, we surround with braces.
\begin{leftfigure}
\begin{lstlisting}
{\itshape Sometimes we want italics}, but only sometimes.
\end{lstlisting}
\end{leftfigure}
is set as
\begin{leftfigure}
\lm {\itshape Sometimes we want italics}, but only sometimes.
\end{leftfigure}
The text within the braces forms a \introduce{group},
and all commands issued inside the group only take effect until it ends.
Environments also form implicit groups:
\begin{leftfigure}
\begin{lstlisting}
\begin{quote}
\itshape If I italicize a quote, the following text will
use upright type again.
\end{quote}
See? Back to normal.
\end{lstlisting}
\end{leftfigure}
produces
\begin{leftfigure}
\lm
\begin{quote}
\itshape If I italicize a quote, the following text will
use upright type again.
\end{quote}
See? Back to normal.
\end{leftfigure}
One other use of groups is to get around the spacing oddities of zero-argument
commands: some prefer \verb|{\LaTeX}| over \verb|\LaTeX{}|.

\chapter{Document Structure}
\label{structure}

Every \LaTeX{} document is different,
but all share a few common elements.

\section{Packages and the preamble}
In the last chapter, you saw:
\begin{leftfigure}
\begin{lstlisting}
\documentclass{article}

\begin{document}
Hello, World!
\end{document}
\end{lstlisting}
\end{leftfigure}
The space between the \verb|\documentclass| command and the start of the
\texttt{document} environment is called the \introduce{preamble}.
Here, we perform any setup we need---such as importing packages
and defining commands---to control how our document
will look.
As mentioned briefly in \chapref{installation},
\introduce{packages} are bits of code that modify your document
in interesting ways.

To import a package, add a \verb|\usepackage| command to the preamble,
with the package's name as the argument.
As a simple example, let's make a document with the \texttt{metalogo}
package, which adds \verb|\LuaLaTeX| and \verb|\XeLaTeX| commands.
\begin{leftfigure}
\begin{lstlisting}
\documentclass{article}

\usepackage{metalogo}

\begin{document}
\XeLaTeX{} and \LuaLaTeX{} are neat.
\end{document}
\end{lstlisting}
\end{leftfigure}
\begin{samepage}
should get you a \textsc{pdf} that reads
\begin{leftfigure}
\lm \XeLaTeX{} and \LuaLaTeX{} are neat.
\end{leftfigure}
\end{samepage}
Like other commands, \verb|\usepackage| accepts optional arguments.
The \texttt{geometry} package, for instance,
takes your desired paper size and margins.
For \acronym{us} letter paper with one-inch margins, you type:\footnote{Notice
that command arguments can be spaced in whatever way is most convenient to you,
so long as there are no empty lines between arguments.}
\begin{leftfigure}
\begin{lstlisting}
\usepackage[letterpaper,
            left=1in, right=1in, top=1in, bottom=1in
           ]{geometry}
\end{lstlisting}
\end{leftfigure}


The packages in your \LaTeX{} distribution come from the Comprehensive \TeX{}
Archive Network, or \acronym{ctan},\punckern\footnote{Curious readers may
be wondering what \TeX{} is, and how it differs from \LaTeX.
The short version is that \TeX{} is the typesetting system that \LaTeX{}
is built on top of---the latter is a framework of commands for the former.
(For example, \texttt{\textbackslash documentclass} and friends are provided by
\LaTeX{}, but the \TeX{} engine is what's actually laying out your document.)
The long version is at the back of this book as Appendix~\ref{history}.
We won't discuss how to use plain \TeX{} here. That's for another book---the
\TeX book.}
at \https{ctan.org}.
The manuals for packages are also found there,
so it should be your first stop when learning how to use one.

\section{Titles, sections, and tables of contents}

Authors often create hierarchy to help readers navigate their work.
\LaTeX{} provides seven different commands to break apart your document:
\verb|\part|, \verb|\chapter|, \verb|\section|, \verb|\subsection|,
\verb|\subsubsection|, \verb|\paragraph|, and \verb|\subparagraph|.
To use one, issue the command where you want that section to start,
using the section's name as the argument.
For example,
\begin{leftfigure}
\begin{lstlisting}
\documentclass{book}

\begin{document}
\chapter{The Start}
This is a very short chapter in a very short book.

\chapter{The End}
Is the book over yet?

\section{No!}
There's some more we must do before we go.

\section{Yes!}
Goodbye!
\end{document}
\end{lstlisting}
\end{leftfigure}
Some levels may not be available depending on the document class
you've chosen. Parts and chapters, for example, only appear in books.
And don't go too crazy with these commands.
Most works only need a few levels of hierarchy.

Sections\footnote{By this I mean all levels, not just
\texttt{\textbackslash section}.}
are automatically numbered---for example,
the title of this chapter was produced with \verb|\chapter{Document Structure}|,
and \LaTeX{} figured out that it was chapter~\ref{structure}.
You can also have \LaTeX{} build you a table of contents
with the \verb|\tableofcontents| command.

\section{What next?}

As promised, this book isn't a comprehensive reference,
but it \emph{will} point you to places where you can learn more.
We'll wrap up most chapters with a list of related topics you could
explore next.

Consider learning how to:
\begin{itemize}
\item Automatically start your document with its title, the author's name,
    and the date using \verb|\maketitle|.
\item Control section numbering with \verb|\setcounter{secnumdepth}|
and ``starred'' section commands, e.g., \verb|\subsection*{foo}|.
\item Create auto-updating cross-references with \verb|\label| and \verb|\ref|.
\item Use KOMA~Script, a set of document classes and packages
that make it easy to customize nearly every aspect of your document,
from section heading fonts to footnotes.
\item Include images using the \texttt{graphicx} package.
\item Add hyperlinks to your \acronym{pdf} using the \texttt{hyperref} package.
\item Split large documents into multiple files using \verb|\input|.
\end{itemize}

\chapter{Formatting Text}
\label{formatting}

\section{Emphasis}

Sometimes you need some extra punch to get your point across.
The simplest way to emphasize text in \LaTeX{} is with the \verb|\emph| command,
which \emph{italicizes} its argument:
\begin{leftfigure}
\begin{lstlisting}
\emph{Oh my!}
\end{lstlisting}
\end{leftfigure}
gives us
\begin{leftfigure}
\lm \emph{Oh my!}
\end{leftfigure}
We have other tools at our disposal:
\begin{leftfigure}
\begin{lstlisting}
We can also use \textbf{boldface} or \textsc{small caps}.
\end{lstlisting}
\end{leftfigure}
producing
\begin{leftfigure}
\lm%
We can also use \textbf{boldface} or \textsc{small caps}.
\end{leftfigure}
Be judicious when you use emphasis, especially boldface,
which excels at drawing the reader's attention away from everything around it.
\textbf{Too much is distracting.}

\section{Meeting the whole (type) family}

Boldface and italics are just a few of the many styles you can use.
A (mostly) complete list follows:
\begin{flushleftfigure}
\lm%
\begin{tabularx}{0.9\textwidth}{l|l|l}
{\normalfont Command} & {\normalfont Alternative} & {\normalfont Style} \\
\hline
\texttt{\textbackslash textnormal\{...\}} & \texttt{\{\textbackslash normalfont ...\}} & the default \\
\texttt{\textbackslash emph\{...\}} & \texttt{\{\textbackslash em ...\}} & \emph{emphasis, typically italics} \\
\texttt{\textbackslash textrm\{...\}} & \texttt{\{\textbackslash rmfamily ...\}} & roman (serif) type \\
\texttt{\textbackslash textsf\{...\}} & \texttt{\{\textbackslash sffamily ...\}} & {\fontspec{Latin Modern Sans}sans serif type} \\
\texttt{\textbackslash texttt\{...\}} & \texttt{\{\textbackslash ttfamily ...\}} & {\fontspec{Latin Modern Mono}teletype (monospaced)} \\
\texttt{\textbackslash textit\{...\}} & \texttt{\{\textbackslash itshape ...\}} & \textit{italics} \\
% WTF: LuaTeX font loading doesn't seem to know what to do with Latin Modern Roman Slant
\texttt{\textbackslash textsl\{...\}} & \texttt{\{\textbackslash slshape ...\}} & {\fontspec{lmromanslant10-regular}slanted, or oblique type} \\
\texttt{\textbackslash textsc\{...\}} & \texttt{\{\textbackslash scshape ...\}} & \textsc{Small Capitals} \\
\texttt{\textbackslash textbf\{...\}} & \texttt{\{\textbackslash bfseries ...\}} & \textbf{boldface} \\
\end{tabularx}
\end{flushleftfigure}
Prefer the first form, which takes the text to format as an argument,
over the second, which affects the group it is issued in.
The former automatically improves spacing around
the formatted text. For example,
\textit{italic type} amidst upright type should be followed
by a slight amount of additional space, called an ``italic correction''\quotekern.
The latter is your only option
when formatting multiple paragraphs
or defining the style of other commands.\punckern\footnote{%
For instance, this book's section headers are styled with
\texttt{\textbackslash Large\allowbreak\textbackslash itshape}.}

\section{Sizes}

The font size of \introduce{body text}---that is, your main content---is
usually ten points,\punckern\footnote{The standard digital publishing point,
sometimes called the PostScript point, is \otffrac{1}{72} of an inch.
\LaTeX{}, for historical reasons, defines its point (\texttt{pt})
as \otffrac{100}{7227} of an inch
and the former as ``big points''\quotekern, or \texttt{bp}.
Use whichever you would like.}
but can be adjusted by passing arguments to
\verb|\documentclass|.\punckern\footnote{Stock \LaTeX{} classes accept
\texttt{10pt}, \texttt{11pt}, or \texttt{12pt} as optional arguments.
KOMA~Script classes accept arbitrary sizes with
\monobox{fontsize=<size>}.}
To scale text relative to this default size, use the following commands:
\begin{flushleftfigure}
\lm%
\renewcommand{\arraystretch}{1.1}%
\begin{tabular}{l l}
\texttt{\textbackslash tiny} & \tiny Example Text \\
\texttt{\textbackslash scriptsize} & \scriptsize Example Text \\
\texttt{\textbackslash footnotesize} & \footnotesize Example Text \\
\texttt{\textbackslash small} & \small Example Text \\
\texttt{\textbackslash normalsize} & \normalsize Example Text \\
\texttt{\textbackslash large} & \large Example Text \\
\end{tabular}
\end{flushleftfigure}
\clearpage
\begin{flushleftfigure}
\lm%
\begin{tabular}{l l}
\texttt{\textbackslash Large} & \Large Example Text \\
\texttt{\textbackslash LARGE} & \LARGE Example Text \\
\texttt{\textbackslash huge} & \huge Example Text \\
\texttt{\textbackslash Huge} & \Huge Example Text \\
\end{tabular}
\end{flushleftfigure}
If you look carefully, you will find some subtleties at play here.
\LaTeX's default type family, Latin Modern,
comes in several \introduce{optical sizes}.
Smaller fonts aren't just shrunken versions of their big siblings---they
have thicker strokes, exaggerated features,
and more generous spacing to improve legibility at their size.
\begin{leftfigure}
\fontspec{lmroman5-regular} If I make 5 point type
\lm the same height as 11 point type,
you can easily spot the differences.
\end{leftfigure}
Back when fonts were made out of metal, multiple optical sizes were standard.
But many digital fonts only have one,
since each optical size requires a great deal of careful
design.\punckern\footnote{If you have typefaces with multiple optical sizes,
\LuaLaTeX{} and \XeLaTeX{} can make good use of them!
See \chapref{fonts} for more on font selection.}

Points and optical sizes don't tell the whole story.
Each typeface has different proportions which affect its perceived size.
(Compare Garamond, {\fontspec{Latin Modern Roman} Latin Modern},
{\fontspec{Futura-Boo}Futura},
and {\fontspec{NHaasGroteskDSPro-45Lt}\addfontfeature{LetterSpace=3}Helvetica}, all at 11 points.)
Shown below are some common terms:
\begin{centerfigure}
\includegraphics[keepaspectratio,width=0.7\textwidth]{heights.png}

\captionof{figure}{Type sits on the \introduce{baseline},
rises to its \introduce{ascender height},
and drops to its \introduce{descender height}.
The \introduce{cap height} refers to the size of uppercase letters,
and the \introduce{x-height} refers to the size of lowercase letters.}
% For size reference:
%{\sffamily\fontsize{8pt}{8pt}\selectfont This is 8-point text.}
\end{centerfigure}

If the previous commands don't give you a size you need,
you can create custom ones with \verb|\fontsize|,
which takes both a text size and a
distance between baselines.
This must be followed with \verb|\selectfont| to take effect.
For example, \texttt{\textbackslash fontsize\{30pt\}\allowbreak\{30pt\}%
\allowbreak\textbackslash selectfont}
produces
\begin{leftfigure}
\lm
\fontsize{30pt}{30pt}\selectfont
large type with no \\
additional space \\
between lines
\end{leftfigure}
{\fontsize{11pt}{11pt}\selectfont
Note how without some extra space,
or \introduce{leading},\punckern\footnote{This term comes from the days of
metal type, when strips of lead or brass were inserted
between lines to space them out.\punckern\endnote{Jan Middendorp, \textit{Shaping Text}
(Amsterdam, 2014), 71}}
descenders from one line almost collide with ascenders and capitals on
the next.
Leading is important---without it, blocks of text become uncomfortable to
read, especially at normal body sizes.\par}
Let your type breathe!\footnote{For a discussion of how much leading
to use, see \textit{Practical Typography},
as mentioned in Appendix~\ref{resources}.}

\exercises{}
\begin{itemize}
\item Learn how to underline text with the \texttt{ulem}
    package.\punckern\footnote{Other typographical tools---like italics,
    boldface, and small caps---are generally preferable to underlining,
    but it has its uses.}
\item Use KOMA~Script to change the size and style of your section headings.
\item Learn the difference between italic and oblique type.
\item Change the default text style
    (used by \verb|\textnormal| and \verb|\normalfont|) by redefining
    \verb|\familydefault|.
\end{itemize}

\chapter{Punctuation}
\label{punctuation}

Quotes, hyphens, dashes, ellipses, oh my!

While we're at it, explain em and en spaces.

\chapter{Layout}

\section{Justification and alignment}

\LaTeX{} is extraordinarily good at justifying text\footnote{Its
only rival here is professional typesetting software like Adobe's.}
for two main reasons:
\begin{itemize}
\item It automatically hyphenates words, allowing lines to be broken
    in many more places besides the spaces between
    words.\punckern\endnote{Franklin Mark Liang,
    \textit{Word Hy-phen-a-tion by Com-put-er} (Standford, 1983),
    \url{http://www.tug.org/docs/liang/}.}

\item Instead of considering each line individually,
    \LaTeX{} considers all possible line breaks for a given paragraph,
    then chooses the one that will give the best spacing according to its
    rules.\punckern\endnote{Donald E.~Knuth and Michael F.~Plass,
    \textit{Breaking Paragraphs Into Lines} (Stanford, 1981)}
\end{itemize}

\mbox{However, if \LaTeX{} cannot make a line fit, it sometimes gives up,
overflowing text into the margin.}
This can be remedied somewhat by adding
\verb|\emergencystretch=<width>| to the document's preamble.
If a paragraph can't be broken to \LaTeX's satisfaction,
this command allows it to try one last time,
stretching or shrinking the space in each line by up to the provided
width.\punckern\footnote{\LaTeX{} has pretty sane default limits to how much
it stretches and shrinks spacing in a paragraph.
You probably don't want to make \texttt{<width>} larger than an em or two.}

TODO: Sometimes, you just have to give up and rewrite to make it fit better.

TODO: Left and right jusitfy

\section{Lists}

\section{Columns}

\section{Page breaks}

Talk about how \TeX{} goes page at a time.
Whine a bit.

\section{What next?}
\begin{itemize}
\item Control paragraph spacing, either using the relevant
KOMA~Script options, or with the standalone \texttt{parskip} package.
\item Set the page size and margins with the \texttt{geometry} package.
\item Insert horizontal and vertical space with commands like
    \verb|\vspace|, \verb|\hspace|, \verb|\vfill|, \verb|\hfill|,
    \verb|\enspace|, \verb|\quad|, and \verb|\,|\,.
\item Learn what units \LaTeX{} provides for specifying spacing.
    (We've already mentioned a few here, such as
    \texttt{pt}, \texttt{bp}, \texttt{mm}, and \texttt{in}.)
\end{itemize}

\chapter{Mathematics}

\LaTeX is excellent at typesetting mathematics, both inline with body text,
e.g., $x_n^2+y_n^2=r^2$, and as standalone formulas:
\[\sum_{n=0}^{\infty} \frac{f^{(n)} (a)}{n!} (x - a)^n\]
The former can be created with \verb|$...$| or \verb|\(...\)|,
and the latter with \verb|\[...\]|.
Inside these environments, the rules of \LaTeX{} change:
\begin{itemize}
\item Most spaces and line breaks are ignored completely,
    and the engine usually makes spacing decisions for you based on
    typographical conventions for math.
    \verb|$x+y+z$| and \verb|$x + y + z$| both give you $x+y+z$.
\item Empty lines are not allowed---each formula must occupy a single
    ``paragraph''\quotekern.
\item Letters are automatically italicized, as they are assumed to be variables.
\end{itemize}
To return to normal ``text mode'' inside a formula, use the \verb|\text| command.
Other formatting commands mentioned in \chapref{formatting} work as well.
From
\begin{leftfigure}
\begin{lstlisting}
\[\text{fake formulas} = \textbf{annoyed mathematicians}\]
\end{lstlisting}
\end{leftfigure}
we get:
\[\text{fake formulas} = \textbf{annoyed mathematicians}\]

\section{What next?}

This chapter is egregiously short.
Typesetting mathematics is arguably the raison d'être of
\LaTeX,\punckern\footnote{Well, \TeX} but the topic is so broad that giving
it fair coverage would take up half the book.
Here, more so than anywhere else,
you owe it to yourself to find some real references and learn what the system
is capable of.

\chapter{Fonts}
\label{fonts}

Digital type has changed almost entirely in the past thirty years.
Originally, \LaTeX{} used \MF,
a system designed by Donald Knuth specifically for \TeX{}.
As time went on, support for PostScript\footnote{One of
Adobe's original claims to fame,
PostScript is a language for defining and drawing computer graphics,
including type. It remains in use today.} fonts was added.
Today, \LuaLaTeX{} and \XeLaTeX{} offer support for the two formats you're
likely to encounter on your computer:
TrueType and OpenType.\punckern\footnote{Mac versions of \LaTeX{} also support
Apple's \acronym{aat}, but we'll limit ourselves here to the more ubiquitous
formats.}

\begin{description}
\item[TrueType] was developed by Apple and Microsoft in the late 1980s.
    Most of the fonts that come pre-installed on your system are likely
    in this format.
    TrueType files generally end in a \monobox{.ttf} extension.
\item[OpenType] was first released in 1996 by Microsoft and Adobe.
    One major improvement over TrueType is its ability to embed
    multiple versions of glyphs in a single font file.
    We'll see this feature in action through much of this chapter.
    OpenType files generally end in an \monobox{.otf} extension.
\end{description}

\section{Changing typefaces}

By default, \LuaLaTeX{} and \XeLaTeX{} use Latin Modern,
an OpenType rendition of \LaTeX's original type family, Computer Modern.
While Latin Modern is a high-quality set of fonts,
you may want to use others for your document.
This is done through the \texttt{fontspec} package:
\begin{leftfigure}
\begin{lstlisting}
\documentclass{article}

\usepackage{fontspec}
\setmainfont[Ligatures=TeX]{Source Serif Pro}
\setsansfont[Ligatures=TeX]{Source Sans Pro}
\setmonofont{Source Code Pro}

\begin{document}
Hello, Source type family! Neat---no? \\
\sffamily Let's try sans serif! \\
\ttfamily Let's try monospaced!
\end{document}
\end{lstlisting}
\end{leftfigure}
should produce something like:\footnote{Assuming, of course,
that you have Adobe's open-source fonts installed.\punckern\endnote{Adobe's
open-source type is freely available at \url{https://github.com/adobe-fonts}.}}
\begin{leftfigure}
\fontspec[Ligatures=TeX]{Source Serif Pro} Hello, Source type family! Neat---no? \\
\fontspec[Ligatures=TeX]{Source Sans Pro} Let's try sans serif! \\
\fontspec{Source Code Pro} Let's try monospaced!
\end{leftfigure}
The \verb|Ligatures=TeX| option allows you to use the standard ligatures
mentioned in \chapref{punctuation} instead of characters that are
unlikely to be on your keyboard.
% No need repeating ourselves here?
%For example, \verb|---| can be used to create em dashes (—),
%quotes can be typed \verb|``like this''| instead of \verb|“like this”|,
%and so on.
You probably don't want this for your monospaced type, though,
since things set in it---such as code---are usually meant to be displayed
verbatim. You don't want \verb|"Hello!"| to turn into
\verb|“Hello!“|

\section{Selecting font files}

A typical typeface might come packaged as four files to represent its
weights and styles, usually
normal,
\textit{italics},
\textbf{bold}, and
\textit{\textbf{bold italics}}.
\texttt{fontspec} can generally deduce the appropriate file
names given the name of the typeface.\punckern\footnote{This is
one of the places \XeLaTeX{} and \LuaLaTeX{}
differ in a way that's noticeable to the casual user.
The former gernally uses system libraries---such as FontConfig on Linux---to
locate files, given a typeface's name.
The latter has its own font loader,
based on code from FontForge.\punckern\endnote{\textit{\LuaTeX{} Reference}
(February 2017, Version 1.0.4), 10.}
The expected name of a font might differ between the two engines---refer
to the \texttt{fontspec} manual for details.}
However, many typefaces come in more than two weights---some versions of Futura,
for example, come in
{\fontspec[Scale=MatchLowercase]{Futura-Lig}light},
{\fontspec[Scale=MatchLowercase]{Futura-Boo}book},
{\fontspec[Scale=MatchLowercase]{Futura-Med}medium},
{\fontspec[Scale=MatchLowercase]{Futura-Dem}demi},
{\fontspec[Scale=MatchLowercase]{Futura-Bol}bold}, and
{\fontspec[Scale=MatchLowercase]{Futura-ExtBol}extra bold}.
Sometimes
{\fontspec[Scale=MatchLowercase]{FuturaSc-Boo}\textsc{small caps}}
are stored as separate files as well.\punckern\footnote{OpenType allows
small caps to be placed in the same file(s) as the other glyphs.
If your font supports this, you don't need to do anything---\texttt{fontspec}
will dutifully switch to them whenever you use
\monobox{\textbackslash textsc} or \monobox{\textbackslash scshape}.
But for TrueType, and for OpenType fonts that don't take advantage of this
feature, you'll have to load a separate file as shown here.}

We might want to hand-pick weights to achieve a certain look or better match the
weights of other fonts in our document.\punckern\footnote{Consider how much
better {\fontspec[Scale=MatchLowercase]{Futura-Boo}the book weight} of Futura
blends in with the surrounding text compared to
{\fontspec[Scale=MatchLowercase]{Futura-Lig}light}
or
{\fontspec[Scale=MatchLowercase]{Futura-Med}medium}.}
Continuing to use Futura as an example,
say we want to use the ``book'' weight as our default
and ``demi'' for bold.
Assuming the font files are named:
\begin{itemize}
\item \monobox{Futura-Boo} for our
    {\fontspec[Scale=MatchLowercase]{Futura-Boo}upright book weight}
\item \monobox{Futura-BooObl} for our
    {\fontspec[Scale=MatchLowercase]{Futura-BooObl}oblique book weight}
\item \monobox{FuturaSC-Boo} for
    {\fontspec[Scale=MatchLowercase]{FuturaSC-Boo}small caps, book weight}
\item \monobox{Futura-Dem} for
    {\fontspec[Scale=MatchLowercase]{Futura-Dem}upright demi(bold)}
\item \monobox{Futura-DemObl} for
    {\fontspec[Scale=MatchLowercase]{Futura-DemObl}oblique demibold}
\end{itemize}

\newpage
\noindent Our font setup might resemble:
\begin{leftfigure}
\begin{lstlisting}
\usepackage{fontspec}
\setmainfont[
    Ligatures=TeX,
    UprightFont = *-Boo,
    ItalicFont = *-BooObl,
    SmallCapsFont = *SC-Boo,
    BoldFont = *-Dem,
    BoldItalicFont = *-DemObl
]{Futura}
\end{lstlisting}
\end{leftfigure}
Note that instead of typing out \monobox{Futura-Boo},
\monobox{Futura-BooObl}, and so on, we can use \texttt{*} to insert the base name.

\section{Scaling}

Using different typefaces to create a cohesive experience is tricky,
especially since---as mentioned in \chapref{formatting}---different typefaces
might look completely different at the same point size.
\texttt{fontspec} can help a bit here by scaling fonts to match either the
x-height or the cap height of your main font with
\verb|Scale=MatchLowercase| or \verb|Scale=MatchUppercase|,
respectively.\footnote{Another way to sidestep this issue is to user fewer
typefaces in your design. Even just one or two typefaces,
used carefully, can produce amazing results.}


\section{OpenType features}

As mentioned above, one of OpenType's defining features is the ability to store
multiple variations of a typeface's glyphs in a single file and allow users
to switch between them.
All of these features can be specified as an optional argument to
\verb|\setmainfont| and friends.
They can also be set for the current group with
\verb|\addfontfeature|.
Lets's touch on a few common ones.

\subsection{Ligatures}

Many typefaces use \introduce{ligatures}, where multiple characters are combined
into a single glyph.\punckern\footnote{Ligatures fell out
of style somewhat during the 20{\addfontfeature{VerticalPosition=Superior}th}
century due to limitations of printing technology---including early
computer typesetting---and the increased popularity of sans serif typefaces,
which often lack them.
Today they are making a comeback,
thanks in no small part to their support in OpenType.
\textbf{Trivia:} glyphs such as the ampersand
(\,\&\,) and the German Eszett (\,ß\,) evolved
from ligatures.}
OpenType groups ligatures into three categories:
\begin{description}
\item[Standard] ligatures are enabled by default, and remedy spacing problems
    a typeface might otherwise have. Consider the lowercase letters f
    and i.
    In many serif typefaces, these combine
    to form the ligature fi, which avoids awkward spacing between the
    ascender of f and the dot of i
    {\addfontfeature{Ligatures=CommonOff} (\,fi\,).}
    Other common examples in English writing include ff,
    ffi, fl, and ffl.
\item[Discretionary] ligatures, such as
    {\addfontfeature{Ligatures=Discretionary}ct},
    are offered by some fonts.
    They are disabled by default
    but can be enabled with
    \verb|Ligatures=Discretionary|.
\item[Historical] ligatures are ones which have fallen out of common use,
    such as those with a \introduce{long~s} (e.g.,
    {\fontspec[Scale=MatchLowercase, Ligatures={TeX,Historic}]{EB Garamond}ſt}).
    These are also disabled by default
    but can be enabled with \verb|Ligatures=Historic|.
\end{description}
In the likely event that you also want to use \verb|Ligatures=TeX|,
multiple options can be grouped together, e.g.,
\verb|Ligatures={TeX,Discretionary}|.
Ligatures can also be disabled using corresponding \verb|*Off|
options. If you wanted to temporarily disable discretionary ligatures,
\begin{leftfigure}
\begin{lstlisting}
{\addfontfeature{Ligatures=DiscretionaryOff}...}
\end{lstlisting}
\end{leftfigure}
would do the trick.

Some words are arguably typeset better without ligatures---a classic example
is shelfful.\punckern\endnote{Knuth, \textit{The \TeX book},
(Addison-Wesley, 1986), 19.}
You can manually prevent the insertion of ligatures with an empty group,
e.g., \verb|shelf{}ful|,
or use the \texttt{selnolig} package to handle most of these cases automatically.

\subsection{Figures}

When setting figures,\punckern\footnote{\introduce{Figure}
here refers to what some might call a \introduce{numeral} or
\introduce{digit}---i.e., 0, 1, 2, 3, 4, 5, 6, 7, 8, and 9.
Typographers generally prefer the first term to the other two.}
you have two
choices to make: lining versus oldstyle,
and proportional versus tabular.
\introduce{Lining} figures, sometimes called \introduce{titling} figures,
have a similar heights to capital letters:
\begin{leftfigure}
\addfontfeature{Numbers=LowercaseOff}
A B C D 1 2 3 4
\end{leftfigure}
\introduce{Oldstyle}, or \introduce{text} figures,
share more similarities with lowercase letters:
\begin{leftfigure}
Sitting cross-legged on the floor\ldots{} 25 or 6 to 4?
\end{leftfigure}
For body text, either is a fine choice, but oldstyle figures shouldn't
be combined with capital letters:
\begin{leftfigure}
``F-15C'' looks odd, as does ``Best Of 2007''\quotekern.
\end{leftfigure}

{\addfontfeature{Numbers=LowercaseOff}
The terms \introduce{proportional} and \introduce{tabular} refer to spacing.
Tabular figures are set with a uniform width, such that 1 takes up
the same space as 8.
As their name suggests, this is great for tables and other scenarios
where figures must line up with the ones above and below them:}
\begin{leftfigure}
\addfontfeature{Numbers={Tabular,LowercaseOff}}
\begin{tabular}{l|c r}
Item & Qty. & Price \\
\hline
Gadgets & 42 & \$5.37 \\
Widgets & 18 & \$12.76 \\
\end{tabular}
\end{leftfigure}
Proportional figures are the opposite---their spacing is, well\ldots{}
\emph{proportional} to the width of the figure.
They are usually preferable in body text, where 1837
looks a bit nicer than
{\addfontfeature{Numbers=Tabular}1837}.

You can select your figures with \verb|Numbers=| and the following options:
\begin{leftfigure}
\begin{tabular}{l l}
\texttt{Numbers=} & \texttt{Lining / Uppercase} \\
 & \texttt{OldStyle / Lowercase} \\
 & \texttt{Proportional} \\
 & \texttt{Tabular / Monospaced}
\end{tabular}
\end{leftfigure}
As you may have noticed, these traits can be mixed and matched.
You can have proportional lining figures
with \verb|Numbers={Proportional,Lining}|,
or tabular oldstyle ones with \verb|Numbers={Tabular,OldStyle}|.
And, like ligatures, each option has a corresponding \verb|*Off|
variant.\punckern\footnote{This is especially useful since fonts
select figures in different ways.
Consider an example where the default figures are lining
and oldstyle figures are enabled with
\monobox{Numbers=OldStyle}.
To return to lining figures in this example,
\monobox{Numbers=Lining}
doesn't work, but
\monobox{Numbers=OldStyleOff}
does.}

Finally, some fonts provide \introduce{superior and inferior} figures,
which can be used for ordinals
(\otford{1}{st}, \otford{2}{nd} \otford{3}{rd}, \ldots),
fractions (\,\otffrac{25}{624}\,), and so on.
These have the same weight as their full-sized counterparts,
which yields a much better result than shrinking normal figures for use as
subscripts and superscripts.
(Notice how
{\addfontfeature{Numbers=LowercaseOff}%
\mbox{1\textsuperscript{st}},
\mbox{2\textsuperscript{nd}},
\mbox{3\textsuperscript{rd}},
and
\,\mbox{\textsuperscript{25}^^^^2044\textsubscript{624}}%
\,}
are too light and narrow compared to the surrounding type.)

\section{What Next?}
\begin{itemize}
\item Learn about more OpenType features supported through \texttt{fontspec},
such as stylistic sets and alternatives.
\item Experiment with letter spacing---or \introduce{tracking}---with
    the \texttt{LetterSpace} option.
    Extra tracking is unnecessary in most cases,
    but can be useful to make \textsc{small caps}
    a bit more \acronym{readable}.
\end{itemize}

\chapter{Microtypography}
\label{microtype}

\introduce{Microtypography} is the craft of improving a document's legibility
with small, subliminal tweaks.
In other words, it is
\begin{quote}
[\dots]the art of enhancing the appearance and readability of a
document while exhibiting a minimum degree of visual obtrusion.
It is concerned with what happens between or at the margins of characters,
words or lines. Whereas the macro-typographical aspects of a document
(i.e., its layout) are clearly visible even to the untrained eye,
micro-typographical refinements should ideally not even be recognisable.
That is, you may think that a document looks beautiful, but you
might not be able to tell exactly why: good micro-typographic practice tries to
reduce all potential irritations that might disturb a reader.\punckern\endnote{%
R Schlicht,
\textit{The microtype package}
(v2.7a, January 14, 2018), 4}
\end{quote}

In \LaTeX{}, microtypography is controlled with the
\texttt{microtype} package.
Its use is automatic---for the vast majority of documents, you should add
\begin{leftfigure}
\begin{lstlisting}
\usepackage{microtype}
\end{lstlisting}
\end{leftfigure}
to your preamble and carry on---but let's take a brief look at what the package
does.

\section{Character protrusion}

By default, \LaTeX{} justifies lines between perfectly straight
left and right margins.
This is the obvious choice,
but falls victim to an annoying optical illusion:
lines ending in small glyphs---like periods, commas,
or hyphens---seem shorter than lines that
don't.\punckern\footnote{Many other optical illusions come up in typography.
For example, if a circle, a square, and a triangle
of equal heights are placed next to each other,
the circle and triangle look smaller than the square.
For this reason, round or pointed characters (like O and A) must
be made slightly taller than ``flat'' ones (such as H and T) for all
to appear the same height.\punckern\endnote{%
Jost Hochuli, \textit{Detail in typography}
(Éditions~\textsc{b}42, 2015),
18--19}}
\texttt{microtype} compensates by \introduce{protruding} these smaller glyphs
into the margins.

\section{Font expansion}

In order to to help \LaTeX's justification algorithm build paragraphs with
more even spacing and fewer hyphenated lines,
\texttt{microtype} can stretch characters horizontally.
You might think that distorting the type this way would be immediately
noticeable,
but you're reading a book that does so on every page!
This effect, called \introduce{font expansion},
is applied \emph{very} slightly---by default,
character widths are altered by no more than two percent.\punckern\footnote{%
Of course, you can use package options to change this limit,
or disable the feature entirely.}

This feature isn't currently available for \XeLaTeX{}.
You'll need to use \LuaLaTeX{} if you'd like to take advantage of it.

\section{What next?}

As always, see the package manual for ways to tweak these features.
\texttt{microtype} is capable of a few other tricks,
but several only work on older \LaTeX{} engines.\punckern\footnote{i.e., pdf\TeX}
Those we do care about---such as letterspacing---can be handled with
\texttt{fontspec} or other packages.

\chapter{Typographie Internationale}

\begin{centerfigure}
\large%
\fontspec[Ligatures=TeX]{NotoSerif}%
Приве́т

\bigskip
\fontspec[Ligatures=TeX]{NotoSerif-Devanagari}%
नमस्ते
\captionof{figure}{How many characters do you see? How many code points?}
\end{centerfigure}

\chapter{When Good Typesetting Goes Bad}

With luck, you're off to a good start with \LaTeX.
But as with any complicated tool, you'll eventually run into trouble.
Here are some common problems and what you can try to fix them.

\section{Fixing overflow}

\mbox{When \LaTeX{} cannot fit a line into a paragraph with good spacing,
it gives up, overflowing} the line into the margin.
You can sometimes fix this by adding
\texttt{\textbackslash emergencystretch=\allowbreak<width>}
to the document's preamble.
When a paragraph can't be broken to \LaTeX's satisfaction,
this command makes it try one last time,
stretching or shrinking the space in each line by up to the provided
width.\punckern\footnote{\LaTeX{} has pretty sane default limits to how much
it stretches and shrinks spacing in a paragraph.
You probably don't want to make \texttt{<width>} larger than an em or two.}
If that doesn't work, try tweaking the paragraph's wording.
This can be frustrating, but the alternative is for \LaTeX{} to create spacing
that is too loose---where\enspace
words\enspace have\enspace large\quad gaps\quad between\enspace
them---or too tight, where\! words\! are\! uncomfortably\! crammed\! together.

\section{Avoiding widows and orphans}

Typesetters do their best to avoid \introduce{widow} lines---which appear
separated from the rest of their paragraph at the start of the following
page.
They also avoid \introduce{orphan}, or \introduce{club} lines,
which are ``left behind'' on the previous page while the rest of the paragraph
begins the next.
\LaTeX{} takes some effort to avoid these, but unfortunately, the algorithm it
uses to split pages is much more simplistic than the one it uses
to split paragraphs into lines.\punckern\footnote{This is because
1980s computers didn't have enough \acronym{ram} to do so. Seriously:
``The computer doesn't have enough high-speed memory capacity to remember the
contents of several pages,
so \TeX{} simply chooses each page break as best it can, by a process of
`local' rather than `global' optimization.\quotekern''\,\endnote{Knuth,
\textit{The \TeX book}, 110}}
You can increase the ``penalty'' for doing so in order to make \LaTeX{}
try harder to avoid it\footnote{When considering a given layout,
\LaTeX{} assigns penalties, or ``badness''\quotekern,
to anything that arguably makes a document look worse.
It chooses whichever layout it can find with the least badness.}
with:
\begin{leftfigure}
\begin{lstlisting}
\widowpenalty=<penalty>
\clubpenalty=<penalty>
\end{lstlisting}
\end{leftfigure}
\verb|<penalty>| is a value between 0 and 10000.
At their maximum, \LaTeX{} is never allowed to leave orphans or widows,
at \emph{any} cost.
This may cause odd layouts to be chosen,
so be sure to review your final layout when picking large penalties.

\section{Handling syntax errors}
If you confuse \LaTeX{}---say, by issuing commands that don't exist,
or forgetting to end an environment---it will print an
error message,\punckern\footnote{Usually this contains a succinct summary of
the problem and the number of the line(s) it occurred on. Occasionally,
\LaTeX{} gets \emph{really} confused and emits something so cryptic it gives
\cpp{} template errors a run for their money.
As you continue to use \LaTeX, you'll start to get a feel for what sorts of
mistakes cause these rare, but enigmatic messages.}
then drop into an interactive prompt starting with \texttt{?}\,.
Here you can enter various instructions for how to proceed.
Once upon a time, when computers were thousands of times slower and
\LaTeX{} took that much longer to re-run, this was more useful.
Today, we likely just want to quit and try again once we've fixed our document.
You do so by typing \texttt{X}, then pressing Enter.
Better yet, you can tell \LaTeX{} to give up as soon as it finds trouble
by running your engine with the \monobox{-halt-on-error} flag:
\begin{leftfigure}
\begin{lstlisting}
$ lualatex -halt-on-error myDocument.tex
\end{lstlisting}
\end{leftfigure}


\appendix
% Don't put big spaces before appendices
\renewcommand{\chapterheadstartvskip}{\vspace{0.15in}}

\chapter{A Brief History of \texorpdfstring{\LaTeX}{LaTeX}}

\label{history}

Donald Knuth is celebrated among programmers as
the man who coined the term \emph{analysis of algorithms} in 1968
and pioneered many of the computer science fundamentals we use today.
Knuth is perhaps most famous for his ongoing magnum opus,
\textit{The Art of Computer Programming}.

When the first volume of \acronym{taocp} was released that same year,
it was printed the way most books had been since the turn of the century:
with \introduce{hot metal} type.
Each individual letter was cast from molten lead,
then arranged into its line.
These lines were clamped together to form pages,
which were finally inked and pressed against paper.

By March of 1977, Knuth was ready for a second run of \acronym{taocp}, volume~2,
but he was horrified when he received the proofs.
Hot metal typesetting was an expensive, complicated, and time-consuming process,
so publishers had replaced it with phototypesetting,
which works by projecting images of characters onto film.
The new technology, while much cheaper and faster,
didn't provide the quality Knuth
expected.\punckern\endnote{Knuth, \textit{Digital Typography} (Stanford, 1999), 3--5}

The average author would have resigned themselves to this change and moved on,
but Knuth took great pride in his books' appearances,
especially their mathematics.
Around this time, he also discovered the growing field of digital typesetting,
where glyphs are built from tiny dots,
packed together at over 1,000 per inch.
Inspired,
Knuth set off on one of the greatest yak shaves\footnote{Programmers
call seemingly unrelated work needed to solve their main problem
``yak shaving''\quotekern. The phrase is thought to originate from an episode
of \textit{The Ren~\&~Stimpy Show}.\punckern\endnote{``yak shaving''\quotekern,
\textit{The Jargon File},
\href{http://www.catb.org/~esr/jargon/html/Y/yak-shaving.html}%
{\texttt{www.catb.org/\~{}esr/jargon/html/Y/yak-shaving.html}}}}
of all time.
For years, he paused work on his books to create his own
typesetting system.
When the dust settled in 1978, Knuth had the first version of
\TeX.\punckern\footnote{The name ``\TeX{}'' comes from the Greek
{\fontspec[Scale=MatchLowercase]{NotoSerif-Medium}τέχνη},
meaning \introduce{art} or \introduce{craft}.\punckern\endnote{Knuth,
\textit{The \TeX book}, 1}}

It's hard to appreciate how much of a revolution \TeX{} was,
especially looking back from a time where anybody with a copy
of Word can be their own desktop publisher.
Adobe's \acronym{pdf} wouldn't exist for another decade, so Knuth
and his graduate students devised their own device-independent format,
\acronym{dvi}.
Scalable fonts were uncommon, so he created \MF{} to rasterize his glyphs
into dots on the page.
Perhaps most importantly, Knuth and his students designed algorithms
to automatically hyphenate and justify text into
beautifully-typeset paragraphs.\punckern\footnote{These same algorithms went
on to influence the ones Adobe uses in its software today.\punckern\endnote{%
Several sources (\http{www.tug.org/whatis.html},
\https{tug.org/interviews/thanh.html},
\http{www.typophile.com/node/34620})
mention \TeX's influence on the \textit{hz}-program by Peter Karow
and Hermann Zapf, thanks to via Knuth's collaborations with Zapf.
\textit{hz} was later acquired by Adobe and used
when creating InDesign's paragraph formatting systems.}}

\LaTeX{}, short for Lamport~\TeX{}, was later developed by Leslie Lamport
as a set of commands for common document layouts.
It was introduced in 1986 with his guide,
\textit{\LaTeX: A~Document Preparation System}.
Other typesetting systems based on \TeX{} also exist,
the other most popular today being Con\TeX{}t.

Development continues,
both in the form of user-provided packages for \TeX{} and \LaTeX{},
and on improvements to the \TeX{} typesetting program itself.
Currently, there are four versions, or \introduce{engines}:
\begin{description}
\item[\TeX] is the original system by Donald Knuth.
Knuth stopped adding features after version 3.0 in March~1990,
and all subsequent releases have contained only bug fixes.
With each release, the version number asymptotically approaches $\pi$
by adding an additional digit.
The most recent version, 3.14159265, came out in January~2014.

\item[pdf\TeX] is an extension of \TeX{} that provides direct \acronym{pdf}
    output (instead of \TeX's \acronym{dvi}),
    native support for PostScript\footnote{%
    \addfontfeature{Numbers=Uppercase}Specifically, PostScript Type 1}
    and TrueType fonts,
    and micro-typographic features discussed in \chapref{microtype}.
    It was originally developed by
    Hàn Thế Thành
    as part of his PhD thesis
    for Masaryk University in Brno, Czech Republic.\punckern\endnote{%
    Hàn Thế Thành,
    \textit{Micro-typographic extensions to the \TeX{} typesetting system}
    (Masaryk University Brno, October 2000)}

\item[\XeTeX] is a further extension of \TeX{} that adds native support for
    Unicode and OpenType.
    It was originally developed by Jonathan Kew in the early 2000s,
    and gained full cross-platform support and inclusion in the \TeX{} Live
    distribution in 2007.\punckern\endnote{Jonathan Kew,
    ``\XeTeX{} Live''\quotekern, \textit{TUGboat} 29, no.~1 (2007)}

\item[\LuaTeX] is similar to \XeTeX{} in its native Unicode and modern font support.
    It also embeds the Lua scripting language into the engine,
    exposing an interface for package and document authors.
    It first appeared in 2007 and is developed by a core team of
    Hans Hagen, Hartmut Henkel, Taco Hoekwater,
    and Luigi Scarso.\punckern\endnote{\http{www.luatex.org}}
\end{description}

Building \TeX{} today is an\dots{} interesting endeavor.
When it was written in the late 1970s,
there were no large, well-documented open-source projects for students to study,
so Knuth set out to make \TeX{} into one.
As part of this effort, \TeX{} was written in a style he calls
\introduce{literate programming}: opposite most programs---where
documentation is interspersed throughout the code---Knuth wrote \TeX{} as a book,
with the code interspersed between paragraphs.
This mix of English and code is called \texttt{WEB}.\punckern\footnote{Knuth
also released a pair of companion programs named
\texttt{TANGLE} and \texttt{WEAVE}.
The former extracts the book---as \TeX, of course---and the latter
produces \TeX's Pascal source code.}

Unsurprisingly, most modern systems don't have good tooling for the late 1970s
dialect of Pascal that \TeX{} was written in,
so present-day distributions use another program,
\texttt{web2c}, to convert its \texttt{WEB} source into C code.
pdf\TeX{} and \XeTeX{} are built by combining the result with other C
and \cpp{} sources.
Instead of following this complicated process,
the \LuaTeX{} authors hand-translated Knuth's Pascal into C
and have been using the resulting code since 2009.\punckern\endnote{%
Taco Hoekwater, \textit{\LuaTeX{} says goodbye to Pascal}
(MAPS 39, Euro\TeX{} 2009),
\https{www.tug.org/TUGboat/tb30-3/tb96hoekwater-pascal.pdf}}


\setlength\parskip{0.8\baselineskip}
\setlength\parindent{0pt}

\chapter{Additional Resources}
\label{resources}

\section{For \texorpdfstring{\LaTeX}{LaTeX}}

As promised at the start, this book is incomplete.
To keep things short,
entire \LaTeX{} features---like figures, captions, references,
and graphics---haven't been mentioned.
Use some of these resources to fill in the gaps,
or as more thorough references:

\begin{adjustwidth}{1.5em}{0pt}
The \LaTeX{} Wikibook, at \url{https://en.wikibooks.org/wiki/LaTeX}

The \TeX{} Stack Exchange, at \url{https://tex.stackexchange.com/}

\textit{The Not So Short Introduction to \LaTeX}, \\
available at \url{https://www.ctan.org/tex-archive/info/lshort/english/}

The Share\LaTeX{} knowledge base, at \url{https://www.sharelatex.com/learn}

\end{adjustwidth}

\section{For typography}

We've spent most of our time here focusing on questions of \emph{what}
you can do with \LaTeX,
and little on \emph{how} you should use it to create quality typography.
Read on:

\begin{adjustwidth}{1.5em}{0pt}
\textit{Practical Typography}, by Matthew Butterick. \\
Available (for free!) at \url{https://practicaltypography.com}

\textit{Stop Stealing Sheep \& Find Out How Type Works}, by Erik Spikermann

\textit{Thinking With Type}, by Ellen Lupton

\textit{The Elements of Typographic Style}, by Robert Bringhurst

\textit{Detail in Typography}, by Jost Hochuli
\end{adjustwidth}

\backmatter

\setkomafont{chapter}{\Huge\itshape}

{\raggedright
\renewcommand\makeenmark{\theenmark.\enspace}
% Chicago Manual of Style, Notes & Bib style, ish.
% http://www.chicagomanualofstyle.org/tools_citationguide/citation-guide-1.html
\theendnotes
}

% Redefine cleardoublepage so the Colophon doesn't demand a front page.
% From https://tex.stackexchange.com/a/24068/92465
{\let\cleardoublepage\clearpage \chapter{Colophon}}

This guide was typeset with \LuaLaTeX{} in Garamond Premier by Robert Slimbach.
His revival is based on Roman type by
\otford{16}{th} century French
punchcutter Claude Garamond.
Italics are inspired by the work of Garamod's contemporary Robert Granjon.

Monospaced items are set in Matthias Tellen's
\href{https://madmalik.github.io/mononoki/}{\texttt{mononoki}},
a typeface designed to work well on both low-resolution computer monitors
and in high-resolution print.

Captions are set in
\href{http://www.fontbureau.com/NHG/}{\textsf{\small Neue Haas Grotesk}},
a Helvetica restoration by Christian Schwartz.
Other digitizations of the classic Swiss typeface are based on fonts made for
Linotype and phototypesetting machines,
resulting in digital versions with all the compromises and kludges from those
past two generations of printing technology.
Schwartz based his work on Helvetica's original drawings,
producing a design faithful to the original cold metal type.

{\fontspec[Ligatures=TeX, Scale=MatchLowercase]{Futura-Boo}URW Futura}
makes a few guest appearances.
Originally released in 1927 by Paul Renner,
Futura has found itself almost everywhere,
from advertising and political campaigns to the moon.
Douglas Thomas's recent history of the typeface,
\textit{Never Use Futura}, is a fantastic read.

Various bits of non-Latin text are set in
{\fontspec[Ligatures=TeX,Scale=MatchLowercase]{NotoSerif-Regular}Noto},
a type family by Google that covers \emph{every} language
in the Unicode standard.

Finally,
{\lm Latin Modern}---the OpenType version of Knuth's Computer Modern used throughout
the book---as well
as {\fontspec[Scale=MatchUppercase]{TeX Gyre Termes}\TeX{} Gyre Termes}---the
free alternative to Times Roman seen on page \pageref{typography}---are from
the digital type foundry of Grupa Użytkowników Systemu \TeX{},
the Polish \TeX{} Users' Group.
An overview of their excellent work can be found at the following locations:\\
\url{http://www.gust.org.pl/projects/e-foundry/latin-modern} \\
\url{http://www.gust.org.pl/projects/e-foundry/tex-gyre}.


\end{document}
