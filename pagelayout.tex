\chapter{Layout}

\section{Justification and alignment}

\LaTeX{} is extraordinarily good at justifying text.
Instead of considering each line individually---as most word processors and
web browsers do---it examines all possible line breaks for a given paragraph,
then chooses ones that will give the best overall
spacing.\punckern\endnote{Donald E.~Knuth and Michael F.~Plass,
\textit{Breaking Paragraphs Into Lines} (Stanford, 1981)}
Combined with its ability to automatically hyphenate words,
which allows line breaks in many more places,\punckern\endnote{%
Franklin Mark Liang,
\textit{Word Hy-phen-a-tion by Com-put-er} (Standford, 1983),
\url{http://www.tug.org/docs/liang/}.}
this produces some of the best paragraph layout available.

\begin{flushleft}
Sometimes, though, you would prefer text be ``flush left''\quotekern,
with a ragged right side. This can be done by placing text in a
\texttt{flushleft} environment, or adding \verb|\raggedright| to the current
group.
\end{flushleft}

\section{Lists}

\section{Columns}

\section{Page breaks}

Talk about how \TeX{} goes page at a time.
Whine a bit.

\section{What next?}
\begin{itemize}
\item Control paragraph spacing, either using the relevant
KOMA~Script options, or with the standalone \texttt{parskip} package.
\item Set the page size and margins with the \texttt{geometry} package.
\item Insert horizontal and vertical space with commands like
    \verb|\vspace|, \verb|\hspace|, \verb|\vfill|, \verb|\hfill|,
    \verb|\enspace|, \verb|\quad|, and \verb|\,|\,.
\item Learn what units \LaTeX{} provides for specifying spacing.
    (We've already mentioned a few here, such as
    \texttt{pt}, \texttt{bp}, \texttt{mm}, and \texttt{in}.)
\end{itemize}
