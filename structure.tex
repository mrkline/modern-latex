\chapter{Document Structure}

While every \LaTeX{} document is different,
all share a few common elements.

\section{Packages and the preamble}
In the last chapter, you saw:
\begin{leftfigure}
\begin{lstlisting}
\documentclass{article}

\begin{document}
Hello, World!
\end{document}
\end{lstlisting}
\end{leftfigure}
The space between the \verb|\documentclass| and \verb|\begin{document}|
is called the \introduce{preamble}.
Here, we perform any setup we need---such as importing packages
and defining commands---to control how the body of our document
(between \verb|\begin{document}| and \verb|\end{document}|) looks.
As mentioned briefly in Chapter~\ref{installation},
\introduce{packages} are bits of code that can modify your document
in interesting ways.
They can do all sorts of things, like:
\begin{multicols}{2}
\begin{itemize}
\item Define new commands
\item Change fonts
\item Add image files
\item Draw vector art
\item Change document layout
\item Add hyperlinks
\end{itemize}
\end{multicols}

To import a package, add the command \verb|\userpackage{name}|
to the preamble.
As a simple example, let's make a document with the \texttt{metalogo}
package, which adds \verb|\LuaLaTeX| and \verb|\XeLaTeX| commands:
\begin{leftfigure}
\begin{lstlisting}
\documentclass{article}

\usepackage{metalogo}

\begin{document}
\XeLaTeX{} and \LuaLaTeX{} are neat.
\end{document}
\end{lstlisting}
\end{leftfigure}
\begin{samepage}
should build a \textsc{pdf} that reads
\begin{leftfigure}
\lm \XeLaTeX{} and \LuaLaTeX{} are neat.
\end{leftfigure}
\end{samepage}
Like other commands, \verb|\usepackage| can take optional arguments.
The \texttt{geometry} package, for instance,
takes the desired paper size and margins.
For \acronym{us} letter paper with one-inch side margins, you could type:
\begin{leftfigure}
\begin{lstlisting}
\usepackage[letterpaper, left=1in, right=1in]{geometry}
\end{lstlisting}
\end{leftfigure}


Many packages are installed as part of your \LaTeX{} distribution,
and any you can't find there can be found on the Comprehensive \TeX{}
Archive Network, or \acronym{ctan}.\punckern\footnote{Curious readers may
be wondering what \TeX{} is, and how it differs from \LaTeX.
The short version is that \TeX{} is the typesetting system that \LaTeX{}
is built on top of---the latter is a framework of commands for the former.
(For example, \texttt{\textbackslash documentclass} and friends are provided by
\LaTeX{}, but the \TeX{} engine is what's actually laying out your document.)
The long version is at the back of the book, under Appendix~\ref{history}.
We won't discuss how to use plain \TeX{} here---that's for another book.
The \TeX book.}
The manuals for every single package can also be found on \acronym{ctan},
so it should be your first stop when learning how to use a new package.

\section{Titles, sections, and tables of contents}

TBD

\section{What next?}

As promised, this book isn't a comprehensive reference,
but it \emph{will} point you to places where you can learn more.
See Appendix~\ref{resources} for good, general \LaTeX{} resources,
but we'll also wrap up each chapter with some ideas you could look into next.

Consider learning about:
\begin{itemize}
\item KOMA~Script, a set of document classes and packages
that make it easy to customize nearly every aspect of your document,
from section heading fonts to footnotes.
\item Including images using the \texttt{graphicsx} package.
\item Adding hyperlinks to your \acronym{pdf} using the \texttt{hyperref} package.
\end{itemize}
