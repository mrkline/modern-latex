\chapter{Document Structure}
\label{structure}

While every \LaTeX{} document is different,
all share a few common elements.

\section{Packages and the preamble}
In the last chapter, you saw:
\begin{leftfigure}
\begin{lstlisting}
\documentclass{article}

\begin{document}
Hello, World!
\end{document}
\end{lstlisting}
\end{leftfigure}
The space between the \verb|\documentclass| and \verb|\begin{document}|
is called the \introduce{preamble}.
Here, we perform any setup we need---such as importing packages
and defining commands---to control how the body of our document
(between \verb|\begin{document}| and \verb|\end{document}|) looks.
As mentioned briefly in \chapref{installation},
\introduce{packages} are bits of code that can modify your document
in interesting ways.
They can do all sorts of things, like:
\begin{multicols}{2}
\begin{itemize}
\item Define new commands
\item Change fonts
\item Add image files
\item Draw vector art
\item Change document layout
\item Add hyperlinks
\end{itemize}
\end{multicols}

To import a package, add a \verb|\usepackage| command to the preamble,
with the package's name as the argument.
As a simple example, let's make a document with the \texttt{metalogo}
package, which adds \verb|\LuaLaTeX| and \verb|\XeLaTeX| commands:
\begin{leftfigure}
\begin{lstlisting}
\documentclass{article}

\usepackage{metalogo}

\begin{document}
\XeLaTeX{} and \LuaLaTeX{} are neat.
\end{document}
\end{lstlisting}
\end{leftfigure}
\begin{samepage}
should get you a \textsc{pdf} that reads
\begin{leftfigure}
\lm \XeLaTeX{} and \LuaLaTeX{} are neat.
\end{leftfigure}
\end{samepage}
Like other commands, \verb|\usepackage| can take optional arguments.
The \texttt{geometry} package, for instance,
can take your desired paper size and margins.
For \acronym{us} letter paper with one-inch margins, you type:
\begin{leftfigure}
\begin{lstlisting}
\usepackage[letterpaper,
            left=1in, right=1in, top=1in, bottom=1in
           ]{geometry}
\end{lstlisting}
\end{leftfigure}


Many packages are installed as part of your \LaTeX{} distribution.
Any you can't find locally are on the Comprehensive \TeX{}
Archive Network, or \acronym{ctan},\punckern\footnote{Curious readers may
be wondering what \TeX{} is, and how it differs from \LaTeX.
The short version is that \TeX{} is the typesetting system that \LaTeX{}
is built on top of---the latter is a framework of commands for the former.
(For example, \texttt{\textbackslash documentclass} and friends are provided by
\LaTeX{}, but the \TeX{} engine is what's actually laying out your document.)
The long version is at the back of the book, under Appendix~\ref{history}.
We won't discuss how to use plain \TeX{} here. That's for another book---the
\TeX book.}
at \url{https://ctan.org}.
The manuals for every single package can also be found there,
so it should be your first stop when learning how to use a new one.

\section{Titles, sections, and tables of contents}

Authors often create hierarchy to help readers navigate their work.
\LaTeX{} provides seven different commands to break apart your document:
\verb|\part|, \verb|\chapter|, \verb|\section|, \verb|\subsection|,
\verb|\subsubsection|, \verb|\paragraph|, and \verb|\subparagraph|.
To use one, just issue the command in your document where you want that section
to start.
For example,
\begin{leftfigure}
\begin{lstlisting}
\documentclass{book}

\begin{document}
\chapter{The Start}
This is a very short chapter in a very short book.

\chapter{The End}
Is the book over yet?

\section{No!}
There's some more we must do before we go.

\section{Yes!}
Goodbye!
\end{document}
\end{lstlisting}
\end{leftfigure}
Some levels may not be available depending on the document class
you've chosen. Parts and chapters, for example, only appear in books.
And don't go too crazy with these commands.
Most works only need a few levels of hierarchy.

Levels are automatically numbered---for example,
the title of this chapter was produced with \verb|\chapter{Document Structure}|,
and \LaTeX{} automatically determined that it was chapter number \ref{structure}.
You can also have \LaTeX{} automatically generate a table of contents for you
by placing \verb|\tableofcontents| at the start of the document body.

\section{What next?}

As promised, this book isn't a comprehensive reference,
but it \emph{will} point you to places where you can learn more.
See Appendix~\ref{resources} for additional resources.
We'll also wrap up most chapters with a list of related topics you could
explore next.

Consider learning how to:
\begin{itemize}
\item Automatically start your document with its title, the author's name,
    and the date using \verb|\maketitle|.
\item Control section numbering with \verb|\setcounter{secnumdepth}|
and ``starred'' section commands, e.g., \verb|\subsection*{foo}|.
\item Create auto-updating cross-references with \verb|\label| and \verb|\ref|.
\item Use KOMA~Script, a set of document classes and packages
that make it easy to customize nearly every aspect of your document,
from section heading fonts to footnotes.
\item Include images using the \texttt{graphicx} package.
\item Add hyperlinks to your \acronym{pdf} using the \texttt{hyperref} package.
\end{itemize}
