\chapter{Formatting Text}
\label{formatting}

\section{Emphasis}

Occasionally your writing needs some extra punch to get its point across.
\LaTeX{} lets you emphasize text with the \verb|\emph| command.
By default, this \emph{italicizes} its argument:
\begin{leftfigure}
\begin{lstlisting}
\emph{Oh my!}
\end{lstlisting}
\end{leftfigure}
gives us
\begin{leftfigure}
\lm \emph{Oh my!}
\end{leftfigure}
There are other tools at our disposal:
\begin{leftfigure}
\lm We can use \textbf{boldface} or \textsc{small caps} as well.
\end{leftfigure}
with
\begin{leftfigure}
\begin{lstlisting}
We can use \textbf{boldface} or \textsc{small caps} as well.
\end{lstlisting}
\end{leftfigure}
Be judicious in your use of emphasis, especially boldface,
which excels at drawing the reader's attention away from everything around it.
Too much is distracting.

\section{Meeting the whole (type) family}

Italics, boldface, and small caps are just a few of the many styles
available to you.
A (mostly) complete list follows:
\begin{leftfigure}
\lm
\begin{tabular}{l|l|l}
Command & ``Switch'' version & Style \\
\hline
\verb|\textnormal{...}| & \verb|{\normalfont ...}| & the default \\
\verb|\emph{...}| & \verb|{\em ...}| & \emph{emphasis, typically italics} \\
\verb|\textrm{...}| & \verb|{\rmfamily ...}| & Roman (serif) type \\
\verb|\textsf{...}| & \verb|{\sffamily ...}| & {\fontspec{Latin Modern Sans}sans serif type} \\
\verb|\texttt{...}| & \verb|{\ttfamily ...}| & {\fontspec[Ligatures=TeX]{Latin Modern Mono}``teletype'' (monospace)} \\
\verb|\textit{...}| & \verb|{\itshape ...}| & \textit{italics} \\
\verb|\textsl{...}| & \verb|{\slshape ...}| & {\fontspec{Latin Modern Roman Slanted}slanted, or oblique type} \\
\verb|\textsc{...}| & \verb|{\scshape ...}| & \textsc{Small Capitals} \\
\verb|\textbf{...}| & \verb|{\bfseries ...}| & \textbf{boldface} \\
\end{tabular}
\end{leftfigure}
The command forms (which take the text to format as an argument)
should usually be preferred over the switch versions
(which affect the group in which they are issued)
because the former automatically handle any
spacing corrections needed around them.\punckern\footnote{For example,
\textit{italicized text} amidst upright text should be followed
by a slight amount of extra space, called an ``italic correction''\quotekern.}
However, there are cases where the switch is the only option:
\begin{enumerate}
\item When formatting multiple paragraphs \\
(Command arguments cannot contain paragraph breaks.)
\item When using them to redefine the style of other commands.\footnote{%
For example, this book sets the style of section headers with
\monobox{\textbackslash setkomafont\{section\}\{\textbackslash Large\textbackslash itshape\}}.}

\end{enumerate}

\section{Sizes}

The default text size is controlled by your document class.
It is usually ten points,\punckern\footnote{The standard digital publishing
point, sometimes called the PostScript point, is \otffrac{1}{72} of an inch.
\LaTeX{}, for historical reasons, defines its point (\texttt{pt})
as \otffrac{100}{7227} of an inch
and the former as a ``big point''\quotekern, or \texttt{bp}.}
but this can be adjusted by passing additonal arguments to
\verb|\documentclass|.\punckern\footnote{The standard \LaTeX{} classes accept
\texttt{10pt}, \texttt{11pt}, or \texttt{12pt}.
KOMA~Script classes accept arbitrary sizes with
\monobox{fontsize=<size>}.}
To scale text relative to this size, use the following commands:
\begin{leftfigure}
\lm
\renewcommand{\arraystretch}{1.1}
\begin{tabular}{l l}
\verb|\tiny| & \tiny Example Text \\
\verb|\scriptsize| & \scriptsize Example Text \\
\verb|\footnotesize| & \footnotesize Example Text \\
\verb|\small| & \small Example Text \\
\verb|\normalsize| & \normalsize Example Text \\
\verb|\large| & \large Example Text \\
\verb|\Large| & \Large Example Text \\
\verb|\LARGE| & \LARGE Example Text \\
\verb|\huge| & \huge Example Text \\
\verb|\Huge| & \Huge Example Text \\
\end{tabular}
\end{leftfigure}
There is some subtlety here that you may not have noticed.
\LaTeX's default type family, Latin Modern,\punckern\footnote{Latin Modern is
an OpenType rendition of \LaTeX's original type family, Computer Modern.}
comes in multiple \introduce{optical sizes}.
Smaller fonts aren't just shrunken versions of their big siblings---they
have thicker strokes, exaggerated features,
and more generous spacing to improve legibility at their size.
\begin{leftfigure}
\fontspec{lmroman5-regular} If you magnify 5 point type
\fontspec{lmroman10-regular} and place the result next to normal 10 point type,
the differences are immediately noticeable.
\end{leftfigure}
Since this requires much more work from the type designer,
many digital typefaces---even professional ones---lack this
feature.\punckern\footnote{If you are fortunate enough \emph{to} have
a typeface with multiple optical sizes, \XeLaTeX{}
and \LuaLaTeX{} can make good use of them! See \chapref{fonts}
for more on font selection.}

Points and optical sizes don't tell the whole story.
Each font has its own proportions, which can make a huge difference
in perceived size.
Shown below are some common terms:
\begin{centerfigure}
\includegraphics[keepaspectratio,width=0.65\textwidth]{heights.png}

\captionof{figure}{Text sits on the \introduce{baseline},
rises to the \introduce{ascender height},
and drops to the \introduce{descender height}.
The \introduce{cap height} refers to the size of uppercase letters,
and the \introduce{x-height} refers to the size of lowercase letters.}
% For size reference:
%{\sffamily\fontsize{8pt}{8pt}\selectfont This is 8-point text.}
\end{centerfigure}
Different fonts vary wildly in these metrics,
resulting in very different looks at the same point size.
Compare Garamond, {\fontspec{Latin Modern Roman} Latin Modern},
and {\fontspec{NHaasGroteskTXPro-55RG}Helvetica}, all at 10\,pt.
%When multiple {\large sizes} {\footnotesize or} {\sffamily\small fonts}
%are set on the same line, their baselines are aligned.

If the previous commands aren't enough, you can create custom sizes with
\verb|\fontsize|, which takes the desired size and distance between baselines.
\verb|\fontsize| must be followed with \verb|\selectfont| to take effect.
For example, \verb|\fontsize{30pt}{30pt}\selectfont|
gives:
\begin{leftfigure}
\lm
\fontsize{30pt}{30pt}\selectfont
large type with no extra \\
space between lines
\end{leftfigure}
{\fontsize{10pt}{10pt}\selectfont
Note that without extra spacing,
or \introduce{leading},\punckern\footnote{This term comes from the days of
metal type, when strips of lead were inserted
between lines to give them extra spacing.}
descenders from one line nearly collide with ascenders and capitals on
the line below.
Leading is important---without it, blocks of text become uncomfortable to
read, especially at normal body sizes.\par}
Let your type breathe.\punckern\footnote{For a discussion on optimal leading,
see \textit{Practical Typography}, listed in Appendix~\ref{resources}.}

\section{What next?}
\begin{itemize}
\item Learn how to underline text using the \texttt{ulem}
    package.\punckern\footnote{Other typographical tools---like italics,
    boldface, and small caps---are usually preferable to underlining,
    but it has its uses.}
\item Use KOMA~Script to change the size and style of your chapter or section
    headings.
\item Learn the difference between italic and oblique type.
\end{itemize}
