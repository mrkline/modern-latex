\chapter{When Good Typesetting Goes Bad}

With luck, you're off to a good start with \LaTeX.
But as with any complicated tool, you'll eventually run into trouble.
Here are some common problems and what you can try to fix them.

\section{Fixing overflow}

\mbox{When \LaTeX{} cannot fit a line into a paragraph with good spacing,
it gives up, overflowing} the line into the margin.
You can sometimes fix this by adding
\texttt{\textbackslash emergencystretch=\allowbreak<width>}
to the document's preamble.
When a paragraph can't be broken to \LaTeX's satisfaction,
this command makes it try one last time,
stretching or shrinking the space in each line by up to the provided
width.\punckern\footnote{\LaTeX{} has pretty sane default limits to how much
it stretches and shrinks spacing in a paragraph.
You probably don't want to make \texttt{<width>} larger than an em or two.}
If that doesn't work, try tweaking the paragraph's wording.
This can be frustrating, but the alternative is for \LaTeX{} to create spacing
that is too loose---where\enspace
words\enspace have\enspace large\quad gaps\quad between\enspace
them---or too tight, where\! words\! are\! uncomfortably\! crammed\! together.

\section{Avoiding widows and orphans}

Typesetters do their best to avoid \introduce{widow} lines---which appear
separated from the rest of their paragraph at the start of the following
page.
They also avoid \introduce{orphan}, or \introduce{club} lines,
which are ``left behind'' on the previous page while the rest of the paragraph
begins the next.
\LaTeX{} takes some effort to avoid these, but unfortunately, the algorithm it
uses to split pages is much more simplistic than the one it uses
to split paragraphs into lines.\punckern\footnote{This is because
1980s computers didn't have enough \acronym{ram} to do so. Seriously:
``The computer doesn't have enough high-speed memory capacity to remember the
contents of several pages,
so \TeX{} simply chooses each page break as best it can, by a process of
`local' rather than `global' optimization.\quotekern''\,\endnote{Knuth,
\textit{The \TeX book}, 110}}
You can increase the ``penalty'' for doing so in order to make \LaTeX{}
try harder to avoid it\footnote{When considering a given layout,
\LaTeX{} assigns penalties, or ``badness''\quotekern,
to anything that arguably makes a document look worse.
It chooses whichever layout it can find with the least badness.}
with:
\begin{leftfigure}
\begin{lstlisting}
\widowpenalty=<penalty>
\clubpenalty=<penalty>
\end{lstlisting}
\end{leftfigure}
\verb|<penalty>| is a value between 0 and 10000.
At their maximum, \LaTeX{} is never allowed to leave orphans or widows,
at \emph{any} cost.
This may cause odd layouts to be chosen,
so be sure to review your final layout when picking large penalties.

\section{Handling syntax errors}
If you confuse \LaTeX{}---say, by issuing commands that don't exist,
or forgetting to end an environment---it will print an
error message,\punckern\footnote{Usually this contains a succinct summary of
the problem and the number of the line(s) it occurred on. Occasionally,
\LaTeX{} gets \emph{really} confused and emits something so cryptic it gives
\cpp{} template errors a run for their money.
As you continue to use \LaTeX, you'll start to get a feel for what sorts of
mistakes cause these rare, but enigmatic messages.}
then drop into an interactive prompt starting with \texttt{?}\,.
Here you can enter various instructions for how to proceed.
Once upon a time, when computers were thousands of times slower and
\LaTeX{} took that much longer to re-run, this was more useful.
Today, we likely just want to quit and try again once we've fixed our document.
You do so by typing \texttt{X}, then pressing Enter.
Better yet, you can tell \LaTeX{} to give up as soon as it finds trouble
by running your engine with the \monobox{-halt-on-error} flag:
\begin{leftfigure}
\begin{lstlisting}
$ lualatex -halt-on-error myDocument.tex
\end{lstlisting}
\end{leftfigure}
