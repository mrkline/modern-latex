\chapter{When Good Typesetting Goes Bad}

Hopefully this book gets you off to a good start with \LaTeX,
but as with any software, you'll eventually run into a bit of trouble.
Here are a few common problems and what you can try to fix them.

\section{Fixing overflow}

\mbox{When \LaTeX{} cannot fit a line into a paragraph with good spacing,
it gives up, overflowing that line}
into the margin.
You can sometimes remedy this by adding
\verb|\emergencystretch=<width>| to the document's preamble.
When a paragraph can't be broken to \LaTeX's satisfaction,
this command has it try one last time,
stretching or shrinking the space in each line by up to the provided
width.\punckern\footnote{\LaTeX{} has pretty sane default limits to how much
it stretches and shrinks spacing in a paragraph.
You probably don't want to make \texttt{<width>} larger than an em or two.}
If that doesn't work, try tweaking the paragraph's wording.
This can be frustrating, but the alternative is for \LaTeX{} to create spacing
that is too loose---where\enspace
words\enspace have\enspace large\quad gaps\enspace between\enspace
them---or too tight, where\! words\! are\! uncomfortably\! crammed\! together.

\section{Handling syntax errors}
If you confuse \LaTeX{}---say, by issuing commands that don't exist or
forgetting to end a group or environment---it will print an
error message,\punckern\footnote{Usually this contains a succinct summary of
the problem and the number of the line(s) it occurred on. Occasionally,
\LaTeX{} gets \emph{really} confused and emits something so cryptic it gives
\cpp{} template metaprogramming errors a run for their money.
As you continue to use \LaTeX, you'll start to get a feel for what sorts of
mistakes cause these rare, but enigmatic messages.}
then drop into an interactive prompt starting with \texttt{?}\,.
Here you can enter various instructions for how to proceed.
Once upon a time, when computers were thousands of times slower and
\LaTeX{} took that much longer to re-run, this was probably more useful.
Today, we likely just want to quit and try again once we've fixed our document.
You do so by typing \texttt{X}, then pressing Enter.
Better yet, you can tell \LaTeX{} to give up as soon as it finds trouble
by running your engine with the \monobox{-halt-on-error} flag:
\begin{leftfigure}
\begin{lstlisting}
$ lualatex -halt-on-error myDocument.tex
\end{lstlisting}
\end{leftfigure}
