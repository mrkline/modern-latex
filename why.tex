\chapter{\texorpdfstring{\LaTeX}{LaTeX}?}

\LaTeX{} is a program\footnote{And a markup language,
and (sort of) a programming language, and a bit of a cult.
But we'll get to all of that.}
for creating written documents, such as papers, presentations,
or even the book you're reading right now.
In some sense, it's an alternative to the likes of Microsoft Word,
Apple Pages, Google Docs, or LibreOffice Writer.

All of these other applications are
\introduce{What You See Is What You Get} \acronym{(wysiwyg)} systems---whatever
you see on screen is what the final document will look like.
This is often incredibly convenient, since many changes can be made
with a few clicks of the mouse,
but it isn't without its downsides.
In a \acronym{wysiwyg} system,
you must constantly concern yourself with both
the \emph{content} of your text and its \emph{layout}.
Keeping the look of a large document consistent is hard---if you increased
the size of one caption, did you make sure to find all the other
captions and do the same?\footnote{This isn't to say that it's impossible
to create nice-looking documents with these tools.
Many of them have gotten much better at automating these sorts of
tasks in recent years.}
If the program formats something in a way you didn't intend,
how difficult is it to change?%\footnote{I spent far too much of my childhood
%fighting with Word about how it wrapped text around images.}

\LaTeX\footnote{Pronounced ``lay-tech'' or ``lah-tex''} is different.
Here, you write your document as a ``plain'' text file,
using \introduce{markup} to specify how things should look.
When you are ready, \LaTeX{} generates your document from this file
based on the rules it was given.
If you've ever done any web development, this is a familiar process.
Just as \acronym{html} and \acronym{css} describe the web page you
want browsers to render, your markup describes the appearance of your
document to \LaTeX{}.

\begin{leftfigure}
\begin{lstlisting}
\LaTeX\footnote{Pronounced ``lay-tech'' or ``lah-tex''} is different.
Here, you write your document as a ``plain'' text file,
using \introduce{markup} to specify how things should look.
When you are ready, \LaTeX{} generates your document from this file
based on the rules it was given.
If you've ever done any web development, this is a familiar process.
Just as \acronym{html} and \acronym{css} describe the web page you
want browsers to render, your markup describes the appearance of your
document to \LaTeX{}.
\end{lstlisting}
\captionof{figure}{The \LaTeX{} markup for the above paragraph}
\end{leftfigure}

This might seem alien to you if you've never worked with a markup system before.
However, it comes with a few advantages:
\begin{enumerate}
\item You can think separately about your document's contents and
    its presentation.
    \LaTeX{} automatically ensures that fonts, sizes, line heights,
    and other layout considerations are consistent according to the rules you set.
\item You can define your own rules, then tweak them to immediately change
    everywhere they're used in your document.
    For example, the \verb|\introduce| and \verb|\acronym| commands you saw above
    are my own creations. The former \introduce{italicizes} text, and
    the latter sets words in \acronym{small caps} with a bit of extra
    \,\textsc{\addfontfeature{LetterSpace=15}letterspacing}\, so the characters
    don't look \textsc{too crowded}.
    If I wake up tomorrow and decide to introduce new terms
    \textbf{\itshape with this look} and set acronyms
    {\small\addfontfeature{LetterSpace=8} LIKE THIS},
    I just change the two lines that define those commands
    and every single place in the book that uses them takes on those new styles.
\item In \LaTeX, the document's structure is immediately visible
    and can be easily copied.
    (Compare this to \acronym{wysiwyg} systems, where it is often not obvious
    how certain formatting was produced
    or how to emulate it.)
\item Since the document source is plain text,
    \begin{itemize}
    \item Document sources can be read and understood with any text editor.
    \item Content can be easily generated programmatically.
    \item Changes can be easily tracked with version control software.
    \end{itemize}
\end{enumerate}

This is all well and good,
but misses the main selling point of \LaTeX.
You should try it because you care about\ldots

\chapter{Typography}
\label{typography}

Modern life is a constant battle for your time---every day,
dozens of ads, apps, emails, sites, and texts fight
for a few short minutes.
To put ideas into the world,
nothing is more important than catching and holding
the attention of your audience.
Typography is a tool to do so---a good design doesn't ``look nice''
only for the sake of art---it draws readers in.\punckern\endnote{Matthew Butterick,
``Typography for Docs''
(presented at the Write The Docs Conference, April 8, 2013),
\url{https://www.youtube.com/watch?v=8J6HuvosP0s}.}
%Matthew Butterick's \textit{Practical Typography} calls it
%``the visual component of the written word.\quotekern''
It sets their expectations and establishes a subliminal ``brand'' for your
work.\punckern\endnote{Erik Spiekermann, ``Type is Visible Language''
(presented at Beyond Tellerrand, Düsseldorf, Germany, May 19--21, 2014), \url{https://www.youtube.com/watch?v=ggQpDu63kk0}.}
It's the \emph{how} of text.
\begin{leftfigure}
\fontspec{TeX Gyre Termes}\fontsize{12pt}{24pt}\selectfont\raggedright
Typography is why this reminds you of the terrible essays
you wrote in school.
Would you like to read an entire book this way?
\end{leftfigure}
It is why
\begin{leftfigure}
\noindent\fontspec{FuturaCon-ExtBol}\Large DO IT LATER
\end{leftfigure}
seems oddly reminiscent of a certain shoe company's advertising,
or how
\begin{center}
\fontspec[Ligatures=TeX, Scale=MatchLowercase]{Futura-Med}
\noindent MEN FROM THE PLANET EARTH \\
FIRST SET FOOT UPON THE MOON \\
JULY 1969, A.~D.
\end{center}
Typography tells people
what you have to say before they read a single word.

This is where \LaTeX{} shines: unless you want to give Adobe large sums
of money for InDesign or InCopy,
you won't find any better typesetting software.
By carefully handling subtle details---like how paragraphs are broken into lines,
or how words are spaced and hyphenated---\LaTeX{} provides high-quality layout
with relatively little effort from you, the author.
Modern versions can also take advantage of new\footnote{By new,
I mean ``from the mid-1990s''\quotekern, but web browsers and desktop publishing
software are only just starting to catch up.} advances in computer typography,
giving you the same tools available to professional designers and publishers.

\chapter{Another Guide?}

One might wonder why the world needs another needs another \LaTeX{} guide.
After all, it's been out for more than 30 years.
A quick Amazon search finds nearly a dozen books on the topic.
There are plenty of great resources online.

Unfortunately, most other guides have two fatal flaws: they are long,
and they are old.
The first is anathema to newcomers---if somebody asks about \LaTeX{},
throwing a 200+ page book at them isn't an encouraging start.
The second matters because of how much the world of computer typesetting has
changed.

When \LaTeX{} was first released in 1986, none of the publishing technologies
we use today existed.
Adobe wouldn't introduce their Portable Document Format for another seven years,
and home computers wouldn't support it ubiquitously for several more.
Digital typography was a new field, and desktop publishing was \emph{just}
getting off the ground.\punckern\endnote{\textit{Graphic Means:
A History of Graphic Design Production}, directed by Briar Levit (2017)}
This shows---badly---in most \LaTeX{} guides.
If you look for instructions to change your document's font,
you'll likely get swamped with bespoke nonsense.\punckern\footnote{%
Take all criticisms of \LaTeX's past here with a grain of
salt. After all, the fact that all of the technology around it became
obsolete---multiple times---is a testament to its staying power.}

The good news is that  \LaTeX{} has improved by leaps and bounds in recent years,
and it's time for a guide that doesn't weigh you down with decades of legacy.
It's also time for a guide that doesn't try to be some comprehensive reference.
After all, you're a smart, resourceful individual who knows how to use a search
engine.
This book will:
\begin{itemize}
\item Teach you the very basics of using \LaTeX.
\item Point you to places where you can learn more.
\item Show you how to use modern technologies like OpenType and micro-typography
    to create professional-quality documents.
\item End promptly thereafter.
\end{itemize}
Let's begin.
